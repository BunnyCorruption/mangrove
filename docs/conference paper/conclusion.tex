\begin{center}
VII. Conclusion
\end{center}
\begin{flushleft}
\setlength{\parindent}{0.125in}
Some of the major successes that come with Mangrove include the fact that this piece of software really is one of a kind. There is no known software available solely for soundscape ecology researchers to both organize and process their datasets. The vision is that in using these graphs, researchers can now publish their findings in a consumable fashion for both other researchers and the general public. Another great feature of Mangrove is the processing capabilities. Before, researchers like our sponsor would have to work in the terminal to run R scripts that had the potential to fail and nullify any progress made. Now, each file is run with live updates to show the user that work is actually happening. Providing an intuitive interface for researchers to use these algorithms to analyze their sound files is a valuable resource in this blooming field.\par
Aside from the technical achievements of Mangrove, through development we have also brought to light some challenges and discrepencies in the field of sound ecology. We have been able to identify problems in the algorithms being used by researchers and are able to propose changes (and have made changes even) to the algorithms to improve the validity of data. Additionally, our machine learning research has shed a great deal of light on the possibilities of machine listening and the implications of that practice as it pertains to soundscape ecology research.\par
\end{flushleft}
