\begin{center}
III. Soundscape Ecology
\end{center}
\begin{flushleft}
\setlength{\parindent}{0.125in}
This project presented a heap of opportunities for research into multiple facets of soundscape ecology, on both the technical and philosophical side. Technologically, we needed to solve a big data issue, where terabytes of information needed to be processed and organized efficiently, in a manner where multiple researchers may be using the same data center. Additionally, processing speeds for sound files using the R language was notoriously slow, so speeding up processing in any way possible was paramount. On the philosophical side, the study of soundscape ecology is relatively new, so the understanding of how the mathematical indices explained before correlate to actual meaning is still limited. Further, the actual validity of these indices when trying to make conclusions as to ecological health is also up for debate.\par

\noindent\textit{A. File Splitting}\par
As the files being used are commonly around hour long, 1.2Gb wav files, the processing time and power needed was drastic, especially when hundreds of files needed to be analyzed. Thus it was considered to split files for both time and memory efficiency. It was concluded that file splitting did indeed improve processing times, by about 70 seconds between 30 minute and 1 minute splits. However upon further research, it was found that splitting files actually changes the output values of the indices. Thus, the validity of those results are questionable. It was decided that file splitting would not be included in processing in the backend in order to preserve data validity.\par

\noindent\textit{B. Performance Enhancements}\par
Go for it Ot\par

\noindent\textit{C. Visualization Research}\par
As each index represents different data in a sound file, it is important to represent that data in a way that makes sense to researchers and for publication. In general, bar charts and line charts are used in the service to represent the data, but each index has its own meaning in its charts. Regardless of index, a line graph representing the index value over time is available. Being able to show index value over time is of great importance to researchers as change over time is great for drawing conclusions when correlating with other forms of research. A great example of this would be our own sponsor Dr. Beever\textquotesingle s research at the Sanford Zoo. There, they are researching whether the man made noise like the train passing by, has any effect on the animal well being.

\end{flushleft}
