\begin{center}
VI. Technologies Used
\end{center}
\begin{flushleft}
  \setlength{\parindent}{0.125in}
The following libraries and frameworks were used heavily in the development of Mangrove.\\

\noindent\textit{A. Electron, ReactJS, Recharts}\par
React is a JavaScript library made for creating user interface components in an easily reusable way. Each React component allows for an HTML mockup with variables that are used as placeholders that we can pass info to. React was essential because we can create HTML templates that are useable with different parameters, for quickly making new and similar pages.\par
Electron is a tool for creating desktop applications using JavaScript, HTML, and CSS. Electron includes cross platform support as well as crash reporting. These features were crucial for this project since our sponsor uses a Mac device and the development team used a variety of Operating Systems.\par
Recharts is the library used for creating the data visualizations from job results. It is built off the D3 library and offers a great API for creating easy to read, high quality graphs using React components. In addition, it offers a wide variety of graph types which is essential to this project as each index being used requires different visual representation. Recharts also allows for interactivity with the graphs themselves, so allowing the user to listen to audio files with the click of a data point is possible.\par

\noindent\textit{B. Express, NodeJS, MongoDB}\par
Express is a JavaScript library for creating a sort of \textquotesingle skeleton\textquotesingle\ of a website. By using Express, we can create an outline of our backend local infrastructure. This is useful for the development of the API we will be using to allow all the working pieces of the project to interact with each other.\par
Using Node and MongoDB, we can interface and query the Mongo database, hereon refered to as the local database. The reason that MongoDB is the best choice for our particular project is that Mongo allows for more dynamic database objects than a SQL database. An example of this would be in the output of the algorithms being run against the sound files. Some of the output is done in lists, and depending on the size of the input files, the number of those lists is arbitrary. This characteristic makes using a SQL database very difficult because we\textquotesingle d need an arbitrary number of rows for each list that is output from the processing. With MongoDB, we can include lists as a field in our database. These lists can be populated in a much more compact way. In addition, during planning, we came up with an efficient way to map parameter research, which we are referring to as jobs, to the inputs they are run on. MongoDB allows reference IDs as fields, so we are going to be implementing an M to N relationship, where a set of inputs, or multiple sets, will be mapped to many different jobs. Configuring the database in this manner allows us to only upload inputs only once and then use it on as many jobs as necessary.\par
Along with higher amounts of flexibility throughout the database, Mongo allows for a hierarchical setup, through the use of discriminators, for each Job/Spec type. Allowing us to create a base Job/Specs, which subclasses can then inherit shared properties from. The hierarchical system allows for quick updates to the system in case of any new Jobs/Specs are added.

\end{flushleft}
