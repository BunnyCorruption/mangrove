\subsection{Data Visualization Research}
With the field of soundscape ecology being a relatively new one, the format of data visualizaions is not yet cemented in the community. In addition to the basic functionality of our service, another goal of this project is to research and define useful visualizations for both researchers and data consumers. Each index analyzes a different thing in nature, so not all forms of data representations will apply. The goal of this section is to shed light on the reasoning behind the decisions to use each index\textquotesingle s respective visualization in the service.

\subsubsection{Outlier Identification}
An important part of analysis on the sound files is to identify when outliers arise. Our sponsor has expressed that being able to see when an outlier occurs and being able to listen to what caused it is important in drawing conclusions and finding interesting bits of information from a data set. Thus, two infographics seem reasonable, those being a timeline or a line chart. Both do relatively the same thing, however a timeline is made specifically for information over time. The only upside to a line graph would be that each point on the graph would represent a sound file, and its respective index output. Being able to represent an output as a small dot on a chart and allowing the user to select that point to see which file caused the outlier seems desirable. Implementing a full fledged sound file analysis section seems a bit out of scope for this project, so a line graph used to represent each sound file and its index output in a data set is the best way to help the user identify outliers in the data.

\subsubsection{NDSI Visualization}
The NDSI index measures the relationship between biophony and anthrophony at a site. This is elaborated on in the Overview of Indices section of this report. Because the NDSI is used for comparing two different variables, the representation for this index is different from that of ACI or ADI. Here, a multi-series chart is best because it compares two variables over time, those being the anthrophony and biophony of each channel on the microphone.

\subsubsection{ACI Visualization}
As ACI is best represented over a period of time, the first infographics that come to mind for representing it are line charts and histograms. A line chart would be able to represent ACI values over time and also represent multiple results across time. This benefits users because it allows them to visually compare results taken at the same time of day, but from different days. Our sponsor mentioned that being able to see if there is a correlation between time of day and when a site is loud or quiet is useful in research. A histogram on the other hand is best fit for representing the distribution of a single variable over time. This would be useful for representing a single result, instead of comparing results over time. Thus, the two visualizations available for ACI are line charts for comparing results, and a histogram for individual results.

\subsubsection{ADI and AEI Visualization}
The ADI index measures the diversity of sounds in a sound file, usually used to draw conclusions as to the diversity of animals at the site. For this index, a single value is used to represent this diversity, the range of which is determined by the user. The algorithms used in this service provide both a right and left channel ADI value where relevant. Thus, a single numerical value representation for single file or dataset analysis is considered the best. However for comparing ADI across data sets, and/or across time, a simple line graph can be used.

\subsubsection{Bioacoustic Index Visualization}
For the Bioacoustic index, the value output by the algorithms is an area value under a curve for both the left and right channels where relevant. This index is mainly used for estimating the number of birds in an area, so a single value representation of these areas seems to be the best form of representation.
