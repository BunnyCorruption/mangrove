\subsection{First Semester Progress and Successes}
\subsubsection{Tools for Success}
Our meeting schedule has changed throughout the semester based on the needs of the project. During the initial design process, we met at least weekly along with daily Discord discussions. These weekly meetings lasted until we all felt we had made a considerable amount of progress on the project. Once most of the design decisions had been made and we decided to start working on the application, we met on an as needed basis. At this point, Discord communication became more frequent. We created Discord channels related to certain areas of the project, like backend, frontend and a channel to discuss ideas and plans for writing this design document. Documents related to the design process are stored in a shared Google Drive folder. Github is now being used for our frontend and backend code, as well as our \LaTeX{} design document. \par
Our team manager has kept up with setting topics and goals before each meeting, as well as keeping track of ideas discussed in meetings. These are also kept in our Google Drive folder. This practice also allows us to keep track of our progress and evaluate if we are behind schedule on any part of the project. \par
We have kept values in mind during design and development. This includes making sure we are considering the needs of our sponsor and future users. \par
The concept of creating 100 ideas has been done throughout our team meetings and Discord communication, with lengthy discussions becoming commonplace in the early stages of this project. This is especially true when thinking about design decisions, many ideas are suggested in our meetings until the team has agreed that the best ones have been found. We have dealt with risks in a similar way to the 100 ideas method. By trying to think ahead to potential risks that may arise in the future, we can plan around them or think of a new idea so that the risk can be avoided. \par
Delegating tasks has been a very useful practice throughout our project so far. This has aided in the design process by splitting up sections of the design document across our team. New ideas were discovered to be discussed in meeting by doing this and we have been able to cover more material by delegating tasks. In development this has also been useful, our team is split up into frontend and backend with subroles within each group, ensuring that no aspect of the project is forgotten. In addition, in writing this paper, more programming has been done in both the front and backend portion in order to provide more information for this report. Namely, the API and analysis visualizations are mostly done. \par

\subsubsection{Milestone Progress}
\paragraph{Phase 0 - Requirement Gathering and Initial Design} \mbox{}\\[\paragraphheaderspace]
The first milestone in this section, gathering requirements has been completed throughout various meetings with our team, sponsor, TAs and professor. Discussions with our sponsor have clarified what it means for our project to be successful and determined which features should be considered stretch goals. In meeting with our professor, the importance of attempting to implement stretch goal features has been discussed. Our team has agreed on pursuing machine learning stretch goals and we have allocated time for this as shown in the milestones section.\par
The aspects of project design, including the database, API/backend and frontend have been through various iterations and have continued to improve throughout the semester. A more detailed account of this process can be found in the Design Iterations section of this document. We have settled on most major design decisions at this time, but will continue to implement design improvements in the next semester if they arise.\par
\paragraph{Phase 1 - Research and Prototyping} \mbox{}\\[\paragraphheaderspace]
We have made a good amount of progress prototyping our server and client applications compared to the milestone dates set at the start of semester one. Progress on the server side application includes some API requests. A request sent through Postman to create a new job will successfully update a local instance of our database. The backend team will continue to develop the requests outlined in the Application Programming Interface section.\par
Prototyping of the client application is also off to a good start. The application runs on Electron in a development environment and the front end team has made progress on pages of the application that are not related to group collaboration. The client prototype application has a navigation panel for each of the pages that have started to be implemented, catalog, job queue and settings.\par
Progress on the catalog page includes job filtering and searching, which is almost completed to function with sample data. On this page, progress on Recharts visualizations for results has also been made for all indices and data structured the way it will be in the database can be viewed in Recharts graphs. By the end of December, these two features should be working together by showing job results of any sample job searched, as well as job comparisons. The job queue page, where a user will start new jobs currently includes UI selection of indices and parameters. The next step on this page is file input selection for jobs and after this is implemented, requests can be sent to the API to make a new job. An outline of all of the input components on the settings page has been completed. Work has started on the functionality of this page, but will require API requests related to users and groups to be completed first.\par
Prototyping of the AWS backend has not started yet. Our team has decided to complete all aspects of the local environment prototypes before moving on to the paid AWS remote version of the prototype.\par
\paragraph{Implementation and Stretch Goals} \mbox{}\\[\paragraphheaderspace]
Implementation of the local desktop application will likely begin in early January. After all aspects of the client application are functioning with sample data, we will start to include API requests to get both applications working with one another. The backend team will continue to create more of the necessary requests for our application while the front end team is working with sample data. It seems reasonable that we will meet the date set in the milestones section, January 25, 2018, for the implementation of the desktop application. Some backend testing has already taken place at this time.\par
Implementation of the AWS backend application may be the only section that we underestimated the time needed to complete in our milestones. Since we have decided to get the local version completely functioning before moving on to this version, it may not be completed by January 25. We will start to focus more on AWS in the start of the second semester and may need more time than anticipated to implement the feature of allowing a user to switch their processing location from local to remote.\par
Stretch goals have not been our primary concern up to this point in order to get all required goals completed first. We do have one designated member of our team researching possible implementations of our stretch goals through machine learning. When the required features are completed, the rest of the team will be able to move their focus to this area and the research already done will serve as a good starting off point for us.\par
