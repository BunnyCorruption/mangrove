\subsection{Plans for Successful Completion of the Project}
In order to achieve the timely completion of our project, a series of principles have been established amongst ourselves. Many of which were inspired by the Senior Design Bootcamp in August.
\begin{itemize}
  \item \textbf{Frequent collaboration/meetings:} By agreeing to meet twice a week, our team members have the opportunity to raise concerns and ensure accountability with each other. And in addition to collaborating in person, we use technologies such as Github, Discord, and Google Drive to keep ourselves in the loop.
  \item \textbf{Setting Goals:} For each meeting, an agenda is written and followed to ensure that all necessary topics are addressed. This includes meetings with TAs and our sponsor, Dr. Beever.
  \item \textbf{Understanding Values:} For each person involved in this project, their exact definitions of success and quality will differ from those of others. Thus, through greater understanding of  our motivations, our ultimate goal should be easier to achieve.
  \item \textbf{``Creating 100 Ideas:''} This concept of continuously questioning the current solution was described during the Bootcamp. By creating ``100 ideas'' our team could gain a better idea of what a successful project should look like. We implement this practice by recording most of our intragroup proposals within meeting notes or on Discord so that they can be built upon later.
  \item \textbf{Understanding Risks:} Any meaningful endeavor will experience setbacks. Some of them so outlandish that planning for them would appear to be sisyphean. But, by planning ahead and attempting to consider potential risks, these event can become less likely to happen during our project.
  \item \textbf{Delegating Tasks:} For a group of five people, a mutual understanding of who is undertaking each task is critical for timely completion. This way, each member can focus on their own tasks and only need to worry about the work of others when we meet or when someone asks for help. The simple promise of \textbf{``do what you say you will do''} is one that is highly valued in our group.
  \item \textbf{Marking Progress:} For long term goals, it is very easy to lose track of the amount of progress that has been made toward achieving them. As a result, our meetings generally involve raising the simple question ``Are we on track?'' and initiating a discussion from there.
  \item \textbf{Establishing Roles:} Each of our members have different skill sets to offer to this project and, as a result, have different tasks they would prefer to work on. To accommodate to this, each member has an established role aligning to specific sectors of the project (e.g. front-end, back-end, and database design). This also benefits members in a manner that was described before: they can focus on their own work without having to worry about the work of others until we meet.
\end{itemize}
