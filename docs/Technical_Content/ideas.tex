\subsection{Project Ideas}
\textit{David} ---
\begin{enumerate}
    \item Using Amazon Web Services\textquotesingle  cloud infrastructure, the project could be run on a serverless architecture. This would both lower the cost of hosting and allow for greater scalability via concurrent Lambda functions that would process the massive archives of sound files necessary in collecting metric data. Note, however, that an analysis must be done on the cost of using Lambdas to process large numbers of sound files, since a high cost per Lambda would translate to a high cost to any researchers using the service.
    \item Allowing for the analysis of sound files using ranges of parameters, as opposed to just static values, would help researchers get a better understanding of the index values for each file or group of files.
\end{enumerate}

\textit{Keith} ---
\begin{enumerate}
    \item In order to unify our codebase into mostly one language, we can use Electron as our front end and MEAN as our backend to have most of our project\textquotesingle s code written in Javascript. This could speed up development time by decreasing the total amount of technologies our team will have to learn.
    \item Perform analysis on software with similar goals such as Kaleidoscope and Raven to understand popular approaches to soundscape ecology
    \item Research the algorithms included within the soundecology package to understand how they work as well as their strengths and weaknesses
\end{enumerate}

\textit{Brita} ---
\begin{enumerate}
    \item Using Shiny as a possible interface for data visualizations in R. We considered the benefits of having an application mostly written in R, but decided to go in a different direction.
    \item Made UI suggestions that would give users the ability to choose which algorithms they would like to run in an analysis. This would speed up the time to analyze files by not running algorithms the user is not interested in at that time.
    \item Helped with the decision to switch to a NoSQL database by looking at examples of some of the data types we would need to store.
\end{enumerate}

\textit{Josh} ---
\begin{enumerate}
    \item Prototyped an interface in Shiny before the decision to move away from Shiny was made. This prototype however sparked ideas for future frontend features that will be included in the final product.
    \item Thought of some team motivating practices for both meetings and learning that will help the team going forward. These included lenient yet practical objective deadlines, resources for education on technologies to be used, and flexible meeting times.
    \item Allow user to create preset index/parameter pairs for analysis. These could be jobs that the researcher frequently runs. This is just a nice quality of life feature for the user.
\end{enumerate}

\textit{Ot} ---
\begin{enumerate}
    \item By allowing researchers to upload data to a collaborative website we can join data from different sections of the world. With this diverse data set we can create maps (heat maps or intensity maps) to demonstrate how sound ecology changes depending on location. An extra feature could be adding a passing of time and seeing how the effects of human noise can affect the biological noise over time.
    \item By analyzing frequency ranges of previous unwanted data we can create a model for preset frequency cutoffs. These presets would help people newer to the field of soundscape ecology choose appropriate cutoffs for analyzing certain sounds.
    \item Using Inverse fourier transforms, we can decompose complex sound waves into their base tones. This might allow for easier analysis of cutoff sounds. This might also make it easier to clean up data for machine listening.
    \item Including a jobs tab in our program to allow monitoring of jobs. This tab will show you the status of your current running job and the queued jobs that are waiting to be processed.
\end{enumerate}
