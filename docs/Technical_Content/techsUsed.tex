\subsection{Technologies Used}

\paragraph{Front End} \mbox{} \\
\textit{Technologies used: Electron, ReactJS, Redux}\par
React is a JavaScript library made for creating user interface components in an easily reusable way. Each React component allows for an HTML mockup with variables that are used as placeholders that we can pass info to. Each component can have parameters, called props, and states, which help keep track of changes in the data flow. This is useful because we can create HTML templates that are useable with different parameters, for quickly making new and similar web pages.\par
React uses a component hierarchy that allows for component props and states to be passed down along to subcomponents, allowing for a responsive and modular page design. An example in this application includes the Catalog page. The table of results available to the user can be filtered based on user defined name, completed or in progress jobs, and the index used in the results. With React, these options exist as states, and when they change, their respective handlers update the component props, which are then passed down the hierarchy of components to create a filtered results table based on the user\textquotesingle s input. In addition to the results table, the analysis view also relies on input from other components, namely the results table, to determine which information to display. When the user selects a result from the results table, the state again changes to supply the correct props to the analysis component, which in turn renders the data visualizations given the selected results.\par
Electron is a tool for creating desktop applications using JavaScript, HTML, and CSS. Electron was originally created to make the Atom IDE but was adopted by other companies to make desktop applications like Spotify. Electron also includes cross platform support as well as crash reporting. The cross platform abilities are crucial for this project because the sponsor uses a Mac device, but even further, cross platform support is all around a good idea.\par

\paragraph{Local Back End} \mbox{}\\
\textit{Technologies used: Express, NodeJS, MongoDB, Mongoose, JavaScript}\par
Express is a JavaScript library for creating a sort of \textquotesingle skeleton\textquotesingle  of a website. By using Express, we can create an outline of our site\textquotesingle s backend local infrastructure. This is useful for the development of the API we will be using to allow all the working pieces of the project to interact with each other.\par
Using Node and MongoDB along with Mongoose, we can interface and query the Mongo database, hereon refered to as the local database. The reason that MongoDB is the best choice for our particular project is that Mongo allows for more dynamic database objects that a SQL database would make a bit more difficult. An example of this would be in the output of the algorithms being run against the sound files. Some of the output is done in lists, and depending on the size of the input files, the number of those lists is arbitrary. This characteristic makes using a SQL database very difficult because we\textquotesingle d need an arbitrary number of rows for each list that is output from the processing. With MongoDB, we can include lists as a field in our database, and there these lists can be populated in a much more compact way. In addition, during planning, we came up with an efficient way to map parameter research, which we are referring to as jobs, to the inputs they are run on. MongoDB allows reference IDs as fields, so we are going to be implenting a M to N relationship, where a set of inputs, or multiple sets, will be mapped to many different jobs. This will allow the user a lot of freedom when choosing what kind of analysis they want to run on their data.\par
Our service\textquotesingle s backend API is written in JavaScript, allowing for JSON objects to be passed back to the client for processing in the various pages. Further API documentation can be found in the Application Programming Interface section of this report.

\paragraph{Remote Back End} \mbox{} \\
\textit{Technologies used: Amazon Web Services (AWS)}\par
Amazon Web Services provide a few great tools that will be useful for our project. First, we are considering using Amazon DynamoDB, a non relational database utilizing NoSQL, much like MongoDB. For this reason, we believe that this would be a great tool for our remote back end implementation so that our two databases can be compatible.\par
Amazon also offers a service called Lambda, where data is processed asynchronously based on developer defined events. These events, in our case, would be when the user creates a job to process. This job, containing sound files, would go through these Lambdas and output the metrics to the DynamoDB implementation. Note that this process is only available should the user opt to process the data using Lambdas. The application being developed will default to allocating processing power to the user\textquotesingle s local machine.\par
In addition to DynamoDB, Amazon also offers a service called S3. S3 is a cloud based object storage system. This storage allows for big data analytics on the stored data, as well as Lambda processing on the data as an event when data is uploaded to a bucket. This technology could potentially be useful for our project, however more research is needed to ensure the best possible implementation.\par
We are planning on using something like DynamoDB for our user and group database for the collaboration side of the project. This collaboration includes every user having a user account, and users being able to set up research groups. Each research group will have the ability to host public or restricted access servers through our site, whether it be through OneDrive, DropBox, Google Drive, or a server like QNAP device. The OneDrive and Google Drive implementations can be done directly through their respective APIs, however allowing the user to give public access to servers like a user QNAP will require the storing of permission credentials by our service. The user will be able to create accounts with access rights on their private servers and give the credentials to our service to allow other users, possibly researchers at another institution, to access those user specified files. We felt this was the best way to handle this because we did not want to force users to create a public directory for \textit{anyone} to access, only those using our service.\par
