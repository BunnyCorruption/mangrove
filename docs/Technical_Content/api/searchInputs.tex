\subsubsection{Search Inputs}
Search for inputs that match a specified description.

\paragraph{GET} \mbox{}\\[\codeheaderspace]
\begin{htmlcode}
https://<base-url>/api/search/inputs
\end{htmlcode}

\paragraph{Header} \mbox{}\\[\tabularheaderspace]
\begingroup
\renewcommand{\arraystretch}{\cellpaddingvertical}
\begin{tabular}{| m{\fieldcolwidth} | m{\typecolwidth} | m{\desccolwidthlg} |}
  \hline
  \reqhead{Field}
  & \reqhead{Type}
  & \reqhead{Description}
  \\ \hline

  \codesnip{Content-Type}
  & String
  & \codesnip{"application/json"}
  \\ \hline
\end{tabular}
\endgroup

\paragraph{Request Body Fields} \mbox{}\\[\tabularheaderspace]
\begingroup
\renewcommand{\arraystretch}{\cellpaddingvertical}
\begin{tabular}{| m{\fieldcolwidth} | m{\typecolwidth} | m{\desccolwidthlg} |}
  \hline
  \reqhead{Field}
  & \reqhead{Type}
  & \reqhead{Description}
  \\ \hline

  \codesnip{filePath}
  & String
  & The path to the input file on the host machine, from the root input folder. Note: This will always be a path to a WAV file.
  \\ \hline

  \codesnip{siteName}
  & String
  & The name of the site in which the input was recorded.
  \\ \hline

  \codesnip{timeRecordedMs}
  & Object
  & The time at which the input audio file was recorded.
  \\ \hline
  \hspace{3mm} \codesnip{min}
  & Number & Minimum value of \codesnip{lat}. \\ \hline
  \hspace{3mm} \codesnip{max}
  & Number & Maximum value of \codesnip{lat}. \\ \hline

  \codesnip{coords}
  & Object
  & The coordinates of the location in which the input was recorded.
  \\ \hline

  \hspace{3mm} \codesnip{lat}
  & Object
  & The latitude of the recording location.
  \\ \hline
  \hspace{6mm} \codesnip{min}
  & Number & Minimum value of \codesnip{lat}. \\ \hline
  \hspace{6mm} \codesnip{max}
  & Number & Maximum value of \codesnip{lat}. \\ \hline

  \hspace{3mm} \codesnip{long}
  & Object
  & The longitude of the recording location.
  \\ \hline
  \hspace{6mm} \codesnip{min}
  & Number & Minimum value of \codesnip{long}. \\ \hline
  \hspace{6mm} \codesnip{max}
  & Number & Maximum value of \codesnip{long}. \\ \hline
\end{tabular}
\endgroup

\paragraph{Example Request Body} \mbox{}\\[\codeheaderspace]
\begin{jsoncode}
{
  "filePath": "fileName.wav",
  "siteName": "UCF",
  "timeRecordedMs": {
    "min": 1504929600000,
    "max": 1505102400000
  },
  "coords": {
    "lat": {
      "min": 28,
      "max": 29
    },
    "long": {
      "min": -82,
      "max": -81
    }
  }
}
\end{jsoncode}

\paragraph{Response Body Fields} \mbox{}\\[\tabularheaderspace]
\begingroup
\renewcommand{\arraystretch}{\cellpaddingvertical}
\begin{tabular}{| m{\fieldcolwidth} | m{\typecolwidth} | m{\desccolwidthlg} |}
  \hline
  \reqhead{Field}
  & \reqhead{Type}
  & \reqhead{Description}
  \\ \hline

  \codesnip{inputId}
  & String
  & A unique identifier for the audio input file.
  \\ \hline

  \codesnip{filePath}
  & String
  & The path to the input file on the host machine, from the root input folder. Note: This will always be a path to a WAV file.
  \\ \hline

  \codesnip{siteName}
  & String
  & The name of the site in which the input was recorded.
  \\ \hline

  \codesnip{timeRecordedMs}
  & Object
  & The time at which the input audio file was recorded.
  \\ \hline

  \codesnip{coords}
  & Object
  & The coordinates of the location in which the input was recorded.
  \\ \hline

  \hspace{3mm} \codesnip{lat}
  & Number
  & The latitude of the recording location.
  \\ \hline

  \hspace{3mm} \codesnip{long}
  & Number
  & The longitude of the recording location.
  \\ \hline
\end{tabular}
\endgroup

\paragraph{Example Response Body} \mbox{}\\[\codeheaderspace]
\begin{jsoncode}
{
  "inputId": "input-1-uuid",
  "filePath": "./path/to/file/fileName.wav",
  "siteName": "UCF Arboretum",
  "timeRecordedMs": 1505016000000,
  "coords": {
    "lat": 28.596238,
    "long": -81.191381
  }
}
\end{jsoncode}
