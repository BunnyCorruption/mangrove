\subsubsection{Search Specifications}
Search for specifications that match a given description.

\paragraph{GET} \mbox{}\\[\codeheaderspace]
\begin{htmlcode}
https://<base-url>/api/search/specs
\end{htmlcode}

\paragraph{Header} \mbox{}\\[\tabularheaderspace]
\begingroup
\renewcommand{\arraystretch}{\cellpaddingvertical}
\begin{tabular}{| m{\fieldcolwidth} | m{\typecolwidth} | m{\desccolwidthlg} |}
  \hline
  \reqhead{Field}
  & \reqhead{Type}
  & \reqhead{Description}
  \\ \hline

  \codesnip{Content-Type}
  & String
  & \codesnip{"application/json"}
  \\ \hline
\end{tabular}
\endgroup

\paragraph{Request Body Fields} \mbox{}\\[\longtableheaderspace]
\begingroup
\renewcommand{\arraystretch}{\cellpaddingvertical}
\begin{longtable}{| m{\fieldcolwidth} | m{\typecolwidth} | m{\metriccolwidth} | m{\desccolwidthsm} |}
  \hline
  \reqhead{Field}
  & \reqhead{Type}
  & \reqhead{Metric}
  & \reqhead{Description}
  \\ \hline

  \codesnip{metric}
  & String
  &
  & Possible values: \codesnip{"aci"}, \codesnip{"adi"}, \codesnip{"aei"}, \codesnip{"bi"}, \codesnip{"ndsi"}
  \\ \hline

  \codesnip{minFreq}
  & Object
  & ACI, BI
  & The minimum frequency to use when calculating the value, in Hertz.
  \\ \hline
  \hspace{3mm} \codesnip{min}
  & Number & & Minimum value of \codesnip{minFreq}. \\ \hline
  \hspace{3mm} \codesnip{max}
  & Number & & Maximum value of \codesnip{minFreq}. \\ \hline

  \codesnip{maxFreq}
  & Object
  & ACI, ADI, AEI, BI
  & The maximum frequency to use when calculating the value, in Hertz.
  \\ \hline
  \hspace{3mm} \codesnip{min}
  & Number & & Minimum value of \codesnip{maxFreq}. \\ \hline
  \hspace{3mm} \codesnip{max}
  & Number & & Maximum value of \codesnip{maxFreq}. \\ \hline

  \codesnip{j}
  & Object
  & ACI
  & The cluster size, in seconds.
  \\ \hline
  \hspace{3mm} \codesnip{min}
  & Number & & Minimum value of \codesnip{j}. \\ \hline
  \hspace{3mm} \codesnip{max}
  & Number & & Maximum value of \codesnip{j}. \\ \hline

  \codesnip{fftW}
  & Object
  & ACI, BI, NDSI
  & The fast fourier transform window.
  \\ \hline
  \hspace{3mm} \codesnip{min}
  & Number & & Minimum value of \codesnip{fftW}. \\ \hline
  \hspace{3mm} \codesnip{max}
  & Number & & Maximum value of \codesnip{fftW}. \\ \hline

  \codesnip{dbThreshold}
  & Object
  & ADI, AEI
  & The threshold.
  \\ \hline
  \hspace{3mm} \codesnip{min}
  & Number & & Minimum value of \codesnip{dbThreshold}. \\ \hline
  \hspace{3mm} \codesnip{max}
  & Number & & Maximum value of \codesnip{dbThreshold}. \\ \hline

  \codesnip{freqStep}
  & Object
  & ADI, AEI
  & The size of frequency bands.
  \\ \hline
  \hspace{3mm} \codesnip{min}
  & Number & & Minimum value of \codesnip{freqStep}. \\ \hline
  \hspace{3mm} \codesnip{max}
  & Number & & Maximum value of \codesnip{freqStep}. \\ \hline

  \codesnip{shannon}
  & Boolean
  & ADI
  & Leave unset to allow for either \codesnip{true} or \codesnip{false}.
  \\ \hline

  \codesnip{anthroMin}
  & Object
  & NDSI
  & The minimum value of the range of frequencies of the anthrophony.
  \\ \hline
  \hspace{3mm} \codesnip{min}
  & Number & & Minimum value of \codesnip{anthroMin}. \\ \hline
  \hspace{3mm} \codesnip{max}
  & Number & & Maximum value of \codesnip{anthroMin}. \\ \hline

  \codesnip{anthroMax}
  & Object
  & NDSI
  & The maximum value of the range of frequencies of the anthrophony.
  \\ \hline
  \hspace{3mm} \codesnip{min}
  & Number & & Minimum value of \codesnip{anthroMax}. \\ \hline
  \hspace{3mm} \codesnip{max}
  & Number & & Maximum value of \codesnip{anthroMax}. \\ \hline

  \codesnip{bioMin}
  & Object
  & NDSI
  & The minimum value of the range of frequencies of the biophony.
  \\ \hline
  \hspace{3mm} \codesnip{min}
  & Number & & Minimum value of \codesnip{bioMin}. \\ \hline
  \hspace{3mm} \codesnip{max}
  & Number & & Maximum value of \codesnip{bioMin}. \\ \hline

  \codesnip{bioMax}
  & Object
  & NDSI
  & The maximum value of the range of frequencies of the biophony.
  \\ \hline
  \hspace{3mm} \codesnip{min}
  & Number & & Minimum value of \codesnip{bioMax}. \\ \hline
  \hspace{3mm} \codesnip{max}
  & Number & & Maximum value of \codesnip{bioMax}. \\ \hline
\end{longtable}
\endgroup

\paragraph{Example Request Body} \mbox{}\\[\codeheaderspace]
\begin{jsoncode}
{
    "metric": "aci",
    "minFreq": {
        "min": 0,
        "max": 0
    },
    "maxFreq": {
        "min": 16000,
        "max": 16000
    },
    "j": {
        "min": 30,
        "max": 30
    },
    "fftW": {
        "min": 10,
        "max": 10
    }
}
\end{jsoncode}

\paragraph{Response Body Fields} \mbox{}\\[\tabularheaderspace]
\begingroup
\renewcommand{\arraystretch}{\cellpaddingvertical}
\begin{tabular}{| m{\fieldcolwidth} | m{\typecolwidth} | m{\desccolwidthlg} |}
  \hline
  \reqhead{Field}
  & \reqhead{Type}
  & \reqhead{Description}
  \\ \hline

  \codesnip{specs}
  & Object[]
  & The list of specifications that satisfy the parameters specified in the request body.
  \\ \hline
\end{tabular}
\endgroup

\paragraph{Example Response Body} \mbox{}\\[\codeheaderspace]
\begin{jsoncode}
{
  "specs": [
    {
      "specId": "spec-a-uuid",
      "metric": "aci",
      "minFreq": 0,
      "maxFreq": 16000,
      "j": 30,
      "fftW": 10
    }
  ]
}
\end{jsoncode}
