\subsubsection{Search Jobs}
Search for jobs that match a specified description, which may contain metadata about the job itself, its input, its job specification, and its results.

\paragraph{GET} \mbox{}\\[\codeheaderspace]
\begin{htmlcode}
https://<base-url>/api/search/jobs
\end{htmlcode}

\paragraph{Header} \mbox{}\\[\tabularheaderspace]
\begingroup
\renewcommand{\arraystretch}{\cellpaddingvertical}
\begin{tabular}{| m{\fieldcolwidth} | m{\typecolwidth} | m{\desccolwidthlg} |}
  \hline
  \reqhead{Field}
  & \reqhead{Type}
  & \reqhead{Description}
  \\ \hline

  \codesnip{Content-Type}
  & String
  & \codesnip{"application/json"}
  \\ \hline
\end{tabular}
\endgroup

\paragraph{URL Parameters} \mbox{}\\[\tabularheaderspace]
\begingroup
\renewcommand{\arraystretch}{\cellpaddingvertical}
\begin{tabular}{| m{\fieldcolwidth} | m{\typecolwidth} | m{\desccolwidthlg} |}
  \hline
  \reqhead{Field}
  & \reqhead{Type}
  & \reqhead{Description}
  \\ \hline

  \codesnip{page}
  & Number
  &
  \\ \hline

  \codesnip{perPage}
  & Number
  &
  \\ \hline
\end{tabular}
\endgroup

\paragraph{Request Body Fields} \mbox{}\\[\longtableheaderspace]
\begingroup
\renewcommand{\arraystretch}{\cellpaddingvertical}
\begin{longtable}{| m{\fieldcolwidth} | m{\typecolwidth} | m{\metriccolwidth} | m{\desccolwidthsm} |}
  \hline
  \reqhead{Field}
  & \reqhead{Type}
  & \reqhead{Metric}
  & \reqhead{Description}
  \\ \hline

  \codesnip{author}
  & String
  &
  & The \codesnip{userId} of the user who made the job request.
  \\ \hline

  \codesnip{creationTimeMs}
  & Number
  &
  & The time of the job's creation, listed in milliseconds since the Unix epoch.
  \\ \hline
  \hspace{3mm} \codesnip{min}
  & Number & & Minimum value of \codesnip{creationTimeMs}. \\ \hline
  \hspace{3mm} \codesnip{max}
  & Number & & Maximum value of \codesnip{creationTimeMs}. \\ \hline

  \codesnip{status}
  & String[]
  &
  & The status of the job. Possible values: \codesnip{"queued"}, \codesnip{"processing"}, \codesnip{"finished"}, \codesnip{"failed"}, \codesnip{"cancelled"}.
  \\ \hline

  \codesnip{inputId}
  & String[]
  &
  & A unique identifier for the audio input file.
  \\ \hline

  \codesnip{filePath}
  & String
  &
  & The path to the input file on the host machine, from the root input folder. Note: This will always be a path to a WAV file.
  \\ \hline

  \codesnip{siteName}
  & String
  &
  & The name of the site in which the input was recorded.
  \\ \hline

  \codesnip{timeRecordedMs}
  & Object
  &
  & The time at which the input audio file was recorded.
  \\ \hline
  \hspace{3mm} \codesnip{min}
  & Number & & Minimum value of \codesnip{lat}. \\ \hline
  \hspace{3mm} \codesnip{max}
  & Number & & Maximum value of \codesnip{lat}. \\ \hline

  \codesnip{coords}
  & Object
  &
  & The coordinates of the location in which the input was recorded.
  \\ \hline

  \hspace{3mm} \codesnip{lat}
  & Object
  &
  & The latitude of the recording location.
  \\ \hline
  \hspace{6mm} \codesnip{min}
  & Number & & Minimum value of \codesnip{lat}. \\ \hline
  \hspace{6mm} \codesnip{max}
  & Number & & Maximum value of \codesnip{lat}. \\ \hline

  \hspace{3mm} \codesnip{long}
  & Object
  &
  & The longitude of the recording location.
  \\ \hline
  \hspace{6mm} \codesnip{min}
  & Number & & Minimum value of \codesnip{long}. \\ \hline
  \hspace{6mm} \codesnip{max}
  & Number & & Maximum value of \codesnip{long}. \\ \hline

  \codesnip{specId}
  & String[]
  &
  & A unique identifier for the specification.
  \\ \hline

  \codesnip{metric}
  & String
  &
  & Possible values: \codesnip{"aci"}, \codesnip{"adi"}, \codesnip{"aei"}, \codesnip{"bi"}, \codesnip{"ndsi"}
  \\ \hline

  \codesnip{minFreq}
  & Object
  & ACI, BI
  & The minimum frequency to use when calculating the value, in Hertz.
  \\ \hline
  \hspace{3mm} \codesnip{min}
  & Number & & Minimum value of \codesnip{minFreq}. \\ \hline
  \hspace{3mm} \codesnip{max}
  & Number & & Maximum value of \codesnip{minFreq}. \\ \hline

  \codesnip{maxFreq}
  & Object
  & ACI, ADI, AEI, BI
  & The maximum frequency to use when calculating the value, in Hertz.
  \\ \hline
  \hspace{3mm} \codesnip{min}
  & Number & & Minimum value of \codesnip{maxFreq}. \\ \hline
  \hspace{3mm} \codesnip{max}
  & Number & & Maximum value of \codesnip{maxFreq}. \\ \hline

  \codesnip{j}
  & Object
  & ACI
  & The cluster size, in seconds.
  \\ \hline
  \hspace{3mm} \codesnip{min}
  & Number & & Minimum value of \codesnip{j}. \\ \hline
  \hspace{3mm} \codesnip{max}
  & Number & & Maximum value of \codesnip{j}. \\ \hline

  \codesnip{fftW}
  & Object
  & ACI, BI, NDSI
  & The fast fourier transform window.
  \\ \hline
  \hspace{3mm} \codesnip{min}
  & Number & & Minimum value of \codesnip{fftW}. \\ \hline
  \hspace{3mm} \codesnip{max}
  & Number & & Maximum value of \codesnip{fftW}. \\ \hline

  \codesnip{dbThreshold}
  & Object
  & ADI, AEI
  & The threshold.
  \\ \hline
  \hspace{3mm} \codesnip{min}
  & Number & & Minimum value of \codesnip{dbThreshold}. \\ \hline
  \hspace{3mm} \codesnip{max}
  & Number & & Maximum value of \codesnip{dbThreshold}. \\ \hline

  \codesnip{freqStep}
  & Object
  & ADI, AEI
  & The size of frequency bands.
  \\ \hline
  \hspace{3mm} \codesnip{min}
  & Number & & Minimum value of \codesnip{freqStep}. \\ \hline
  \hspace{3mm} \codesnip{max}
  & Number & & Maximum value of \codesnip{freqStep}. \\ \hline

  \codesnip{shannon}
  & Boolean
  & ADI
  & Leave unset to allow for either \codesnip{true} or \codesnip{false}.
  \\ \hline

  \codesnip{anthroMin}
  & Object
  & NDSI
  & The minimum value of the range of frequencies of the anthrophony.
  \\ \hline
  \hspace{3mm} \codesnip{min}
  & Number & & Minimum value of \codesnip{anthroMin}. \\ \hline
  \hspace{3mm} \codesnip{max}
  & Number & & Maximum value of \codesnip{anthroMin}. \\ \hline

  \codesnip{anthroMax}
  & Object
  & NDSI
  & The maximum value of the range of frequencies of the anthrophony.
  \\ \hline
  \hspace{3mm} \codesnip{min}
  & Number & & Minimum value of \codesnip{anthroMax}. \\ \hline
  \hspace{3mm} \codesnip{max}
  & Number & & Maximum value of \codesnip{anthroMax}. \\ \hline

  \codesnip{bioMin}
  & Object
  & NDSI
  & The minimum value of the range of frequencies of the biophony.
  \\ \hline
  \hspace{3mm} \codesnip{min}
  & Number & & Minimum value of \codesnip{bioMin}. \\ \hline
  \hspace{3mm} \codesnip{max}
  & Number & & Maximum value of \codesnip{bioMin}. \\ \hline

  \codesnip{bioMax}
  & Object
  & NDSI
  & The maximum value of the range of frequencies of the biophony.
  \\ \hline
  \hspace{3mm} \codesnip{min}
  & Number & & Minimum value of \codesnip{bioMax}. \\ \hline
  \hspace{3mm} \codesnip{max}
  & Number & & Maximum value of \codesnip{bioMax}. \\ \hline

  \codesnip{aciTotAllL}
  & Object
  & ACI
  & The ACI total for the left channel.
  \\ \hline
  \hspace{3mm} \codesnip{min}
  & Number & & Minimum value of \codesnip{aciTotAllL}. \\ \hline
  \hspace{3mm} \codesnip{max}
  & Number & & Maximum value of \codesnip{aciTotAllL}. \\ \hline

  \codesnip{aciTotAllR}
  & Object
  & ACI
  &The ACI total for the right channel.
  \\ \hline
  \hspace{3mm} \codesnip{min}
  & Number & & Minimum value of \codesnip{aciTotAllR}. \\ \hline
  \hspace{3mm} \codesnip{max}
  & Number & & Maximum value of \codesnip{aciTotAllR}. \\ \hline

  \codesnip{aciTotAllLByMin}
  & Object
  & ACI
  & The ACI total for the left channel divided by the number of minutes.
  \\ \hline
  \hspace{3mm} \codesnip{min}
  & Number & & Minimum value of \codesnip{aciTotAllLByMin}. \\ \hline
  \hspace{3mm} \codesnip{max}
  & Number & & Maximum value of \codesnip{aciTotAllLByMin}. \\ \hline

  \codesnip{aciTotAllRByMin}
  & Object
  & ACI
  & The ACI total for the right channel divided by the number of minutes.
  \\ \hline
  \hspace{3mm} \codesnip{min}
  & Number & & Minimum value of \codesnip{aciTotAllRByMin}. \\ \hline
  \hspace{3mm} \codesnip{max}
  & Number & & Maximum value of \codesnip{aciTotAllRByMin}. \\ \hline

  \codesnip{adiL}
  & Object
  & ADI
  & The ADI value for the left channel.
  \\ \hline
  \hspace{3mm} \codesnip{min}
  & Number & & Minimum value of \codesnip{adiL}. \\ \hline
  \hspace{3mm} \codesnip{max}
  & Number & & Maximum value of \codesnip{adiL}. \\ \hline

  \codesnip{adiR}
  & Object
  & ADI
  & The ADI value for the right channel.
  \\ \hline
  \hspace{3mm} \codesnip{min}
  & Number & & Minimum value of \codesnip{adiR}. \\ \hline
  \hspace{3mm} \codesnip{max}
  & Number & & Maximum value of \codesnip{adiR}. \\ \hline

  \codesnip{aeiL}
  & Object
  & AEI
  & The AEI for the left channel.
  \\ \hline
  \hspace{3mm} \codesnip{min}
  & Number & & Minimum value of \codesnip{aeiL}. \\ \hline
  \hspace{3mm} \codesnip{max}
  & Number & & Maximum value of \codesnip{aeiL}. \\ \hline

  \codesnip{aeiR}
  & Object
  & AEI
  & The AEI for the right channel.
  \\ \hline
  \hspace{3mm} \codesnip{min}
  & Number & & Minimum value of \codesnip{aeiR}. \\ \hline
  \hspace{3mm} \codesnip{max}
  & Number & & Maximum value of \codesnip{aeiR}. \\ \hline

  \codesnip{areaL}
  & Object
  & BI
  & The area under the curve for the left channel.
  \\ \hline
  \hspace{3mm} \codesnip{min}
  & Number & & Minimum value of \codesnip{areaL}. \\ \hline
  \hspace{3mm} \codesnip{max}
  & Number & & Maximum value of \codesnip{areaL}. \\ \hline

  \codesnip{areaR}
  & Object
  & BI
  & The area under the curve for the right channel.
  \\ \hline
  \hspace{3mm} \codesnip{min}
  & Number & & Minimum value of \codesnip{areaR}. \\ \hline
  \hspace{3mm} \codesnip{max}
  & Number & & Maximum value of \codesnip{areaR}. \\ \hline

  \codesnip{ndsiL}
  & Object
  & NDSI
  & The NDSI value for the left channel.
  \\ \hline
  \hspace{3mm} \codesnip{min}
  & Number & & Minimum value of \codesnip{ndsiL}. \\ \hline
  \hspace{3mm} \codesnip{max}
  & Number & & Maximum value of \codesnip{ndsiL}. \\ \hline

  \codesnip{ndsiR}
  & Object
  & NDSI
  & The NDSI value for the right channel.
  \\ \hline
  \hspace{3mm} \codesnip{min}
  & Number & & Minimum value of \codesnip{ndsiR}. \\ \hline
  \hspace{3mm} \codesnip{max}
  & Number & & Maximum value of \codesnip{ndsiR}. \\ \hline

  \codesnip{biophonyL}
  & Object
  & NDSI
  & The biophony value for the left channel.
  \\ \hline
  \hspace{3mm} \codesnip{min}
  & Number & & Minimum value of \codesnip{biophonyL}. \\ \hline
  \hspace{3mm} \codesnip{max}
  & Number & & Maximum value of \codesnip{biophonyL}. \\ \hline

  \codesnip{biophonyR}
  & Object
  & NDSI
  & The biophony value for the right channel.
  \\ \hline
  \hspace{3mm} \codesnip{min}
  & Number & & Minimum value of \codesnip{biophonyR}. \\ \hline
  \hspace{3mm} \codesnip{max}
  & Number & & Maximum value of \codesnip{biophonyR}. \\ \hline

  \codesnip{anthrophonyL}
  & Object
  & NDSI
  & The anthrophony value for the left channel.
  \\ \hline
  \hspace{3mm} \codesnip{min}
  & Number & & Minimum value of \codesnip{anthrophonyL}. \\ \hline
  \hspace{3mm} \codesnip{max}
  & Number & & Maximum value of \codesnip{anthrophonyL}. \\ \hline

  \codesnip{anthrophonyR}
  & Object
  & NDSI
  & The anthrophony value for the right channel.
  \\ \hline
  \hspace{3mm} \codesnip{min}
  & Number & & Minimum value of \codesnip{anthrophonyR}. \\ \hline
  \hspace{3mm} \codesnip{max}
  & Number & & Maximum value of \codesnip{anthrophonyR}. \\ \hline
\end{longtable}
\endgroup

\paragraph{Example Request Body} \mbox{}\\[\codeheaderspace]
\begin{jsoncode}
{
  "author": "user-1-uuid",
  "creationTimeMs": {
    "min": 1546232400000,
    "max": 1546405200000
  },
  "status": [
    "finished"
  ],
  "filePath": "fileName.wav",
  "siteName": "UCF",
  "timeRecordedMs": {
    "min": 1504929600000,
    "max": 1505102400000
  },
  "coords": {
    "lat": {
      "min": 28,
      "max": 29
    },
    "long": {
      "min": -82,
      "max": -81
    }
  },
  "metric": "aci",
  "minFreq": {
    "min": 0,
    "max": 0
  },
  "maxFreq": {
    "min": 16000,
    "max": 16000
  },
  "j": {
    "min": 30,
    "max": 30
  },
  "fftW": {
    "min": 10,
    "max": 10
  },
  "aciTotAllL": {
    "min": 9,
    "max": 10
  }
  "aciTotAllLByMin": {
    "min": 25,
    "max": 35
  }
}
\end{jsoncode}

\paragraph{Response Body Fields} \mbox{}\\[\tabularheaderspace]
\begingroup
\renewcommand{\arraystretch}{\cellpaddingvertical}
\begin{tabular}{| m{\fieldcolwidth} | m{\typecolwidth} | m{\desccolwidthlg} |}
  \hline
  \reqhead{Field}
  & \reqhead{Type}
  & \reqhead{Description}
  \\ \hline

  \codesnip{jobs}
  & Object[]
  & The list of jobs that satisfy the parameters specified in the request body.
  \\ \hline
\end{tabular}
\endgroup

\paragraph{Example Response Body} \mbox{}\\[\codeheaderspace]
\begin{jsoncode}
{
  "jobs": [
    {
      "jobId": "job-1a-uuid",
      "author": "user-1-uuid",
      "input": "input-1-uuid",
      "spec": "spec-a-uuid",
      "creationTimeMs": 1546318800000,
      "status": "finished"
    }
  ]
}
\end{jsoncode}

\paragraph{Error Handling} \mbox{}\\[\longtableheaderspace]
\begingroup
\renewcommand{\arraystretch}{\cellpaddingvertical}
\begin{longtable}{| m{\errconditioncol} | m{\errcodecol} | m{\errbodycol} |}
  \hline
  \reqhead{Condition}
  & \multicolumn{2}{|l|}{\reqhead{Response}}
  \\ \hline

  A field was included that was not listed as a valid request body field.
  & 400
  & An object containing a single field, \codesnip{message} (String), identifying the invalid field from the request.
  \\ \hline

  A field contains an invalid value.
  & 400
  & An object containing a single field, \codesnip{message} (String), identifying the field in question and its possible values.
  \\ \hline
\end{longtable}
\endgroup
