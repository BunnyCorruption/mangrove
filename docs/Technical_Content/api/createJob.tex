\subsubsection{Create Job}
Create a job based on an input and a specification. The job will be entered into a queue on the server to be processed for metric results.

\paragraph{PUT} \mbox{}\\[\codeheaderspace]
\begin{htmlcode}
https://<base-url>/api/jobs
\end{htmlcode}

\paragraph{Header} \mbox{}\\[\tabularheaderspace]
\begingroup
\renewcommand{\arraystretch}{\cellpaddingvertical}
\begin{tabular}{| m{\fieldcolwidth} | m{\typecolwidth} | m{\desccolwidthlg} |}
  \hline
  \reqhead{Field}
  & \reqhead{Type}
  & \reqhead{Description}
  \\ \hline

  \codesnip{Content-Type}
  & String
  & \codesnip{"application/json"}
  \\ \hline
\end{tabular}
\endgroup

\paragraph{Request Body Fields} \mbox{}\\[\longtableheaderspace]
\begingroup
\renewcommand{\arraystretch}{\cellpaddingvertical}
\begin{longtable}{| m{\fieldcolwidth} | m{\typecolwidth} | m{\desccolwidthlg} |}
  \hline
  \reqhead{Field}
  & \reqhead{Type}
  & \reqhead{Description}
  \\ \hline

  \codesnip{type}
  & String
  & The type of metric to be run on the job. Possible values: \codesnip{"aci"}, \codesnip{"adi"}, \codesnip{"aei"}, \codesnip{"bi"}, \codesnip{"ndsi"}, \codesnip{"rms"}.
  \\ \hline

  \codesnip{inputId}
  & String
  & The input file (listed by \codesnip{inputId}) to be analyzed for metrics.
  \\ \hline

  \codesnip{specId}
  & String
  & The \codesnip{specId} of the specification to be used to analyze the input.
  \\ \hline
\end{longtable}
\endgroup

\paragraph{Example Request Body} \mbox{}\\[\codeheaderspace]
\begin{jsoncode}
{
  "type": "aci",
  "inputId": "input-1-uuid",
  "specId": "spec-a-uuid"
}
\end{jsoncode}

\paragraph{Response Body Fields} \mbox{}\\[\longtableheaderspace]
\begingroup
\renewcommand{\arraystretch}{\cellpaddingvertical}
\begin{longtable}{| m{\fieldcolwidth} | m{\typecolwidth} | m{\desccolwidthlg} |}
  \hline
  \reqhead{Field}
  & \reqhead{Type}
  & \reqhead{Description}
  \\ \hline

  \codesnip{jobId}
  & String
  & A unique identifier for the job created.
  \\ \hline

  \codesnip{type}
  & String
  & The type of metric to be run on the job. Possible values: \codesnip{"aci"}, \codesnip{"adi"}, \codesnip{"aei"}, \codesnip{"bi"}, \codesnip{"ndsi"}, \codesnip{"rms"}.
  \\ \hline

  \codesnip{input}
  & String
  & The \codesnip{inputId} of the input to be analyzed for metrics.
  \\ \hline

  \codesnip{spec}
  & String
  & The \codesnip{specId} of the spec to be used for the job.
  \\ \hline

  \codesnip{author}
  & String
  & The \codesnip{userId} of the user who made the job request.
  \\ \hline

  \codesnip{creationTimeMs}
  & Number
  & The time of the job's creation, listed in milliseconds since the Unix epoch.
  \\ \hline

  \codesnip{status}
  & String
  & The status of the job. Possible values: \codesnip{"queued"}, \codesnip{"processing"}, \codesnip{"finished"}, \codesnip{"failed"}, \codesnip{"cancelled"}.
  \\ \hline
\end{longtable}
\endgroup

\paragraph{Example Response Body} \mbox{}\\[\codeheaderspace]
\begin{jsoncode}
{
  "jobId": "job-1a-uuid",
  "type": "aci",
  "input": "input-1-uuid",
  "spec": "spec-a-uuid",
  "author": "user-1-uuid",
  "creationTimeMs": 1546318800000,
  "status": "queued"
}
\end{jsoncode}
