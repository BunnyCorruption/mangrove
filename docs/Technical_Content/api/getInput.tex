\subsubsection{Get Input}
Retrieve the details of an existing input by \codesnip{inputId}.

\paragraph{GET} \mbox{}\\[\codeheaderspace]
\begin{htmlcode}
https://<base-url>/inputs/:inputId
\end{htmlcode}

\paragraph{URL Parameters} \mbox{}\\[\longtableheaderspace]
\begingroup
\renewcommand{\arraystretch}{\cellpaddingvertical}
\begin{longtable}{| m{\fieldcolwidth} | m{\typecolwidth} | m{\desccolwidthlg} |}
  \hline
  \tablehead{Field}
  & \tablehead{Type}
  & \tablehead{Description}
  \\ \hline

  \codesnip{inputId}
  & String
  & The \codesnip{inputId} of the input for which details are desired.
  \\ \hline
\end{longtable}
\endgroup

\paragraph{Example Request: GET} \mbox{}\\[\codeheaderspace]
\begin{htmlcode}
https://<base-url>/inputs/input-1-uuid
\end{htmlcode}

\paragraph{Response Body Fields} \mbox{}\\[\longtableheaderspace]
\begingroup
\renewcommand{\arraystretch}{\cellpaddingvertical}
\begin{longtable}{| m{\fieldcolwidth} | m{\typecolwidth} | m{\desccolwidthlg} |}
  \hline
  \tablehead{Field}
  & \tablehead{Type}
  & \tablehead{Description}
  \\ \hline

  \codesnip{inputId}
  & String
  & A unique identifier for the audio input file.
  \\ \hline

  \codesnip{path}
  & String
  & The path to the input file on the host machine, from the root input folder. Note: This will always be a path to a WAV file.
  \\ \hline

  \codesnip{siteName}
  & String
  & The name of the site in which the input was recorded.
  \\ \hline

  \codesnip{recordTimeMs}
  & Object
  & The time at which the input audio file was recorded, listed in milliseconds since the Unix epoch.
  \\ \hline

  \codesnip{coords}
  & Object
  & The coordinates of the location in which the input was recorded.
  \\ \hline

  \hspace{3mm} \codesnip{lat}
  & Number
  & The latitude of the recording location.
  \\ \hline

  \hspace{3mm} \codesnip{long}
  & Number
  & The longitude of the recording location.
  \\ \hline
\end{longtable}
\endgroup

\paragraph{Example Response Body} \mbox{}\\[\codeheaderspace]
\begin{jsoncode}
{
  "inputId": "input-1-uuid",
  "path": "path/to/file/arboretum.wav",
  "siteName": "UCF Arboretum",
  "recordTimeMs": 1505016000000,
  "coords": {
    "lat": 28.596238,
    "long": -81.191381
  }
}
\end{jsoncode}

\paragraph{Error Handling} \mbox{}\\[\longtableheaderspace]
\begingroup
\renewcommand{\arraystretch}{\cellpaddingvertical}
\begin{longtable}{| m{\errconditioncol} | m{\errcodecol} | m{\errbodycol} |}
  \hline
  \tablehead{Condition}
  & \multicolumn{2}{|l|}{\tablehead{Response}}
  \\ \hline

  The requested input does not exist.
  & 404
  & An object containing a single field, \codesnip{message} (String), stating that a input with the \codesnip{inputId} provided in the request was not found.
  \\ \hline
\end{longtable}
\endgroup
