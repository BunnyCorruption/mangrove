\subsection{Creating Flexible Visualizations With Recharts}
Using the Recharts framework, expanded on in the technologies used section of this paper, we can create flexible line charts, bar graphs, and more to help explain the user\textquotesingle s data and draw conclusions. This section will cover some of the code used in this service to ensure that our code base is as flexible as possible while still creating useful visualizations.\\

\noindent Common components used in each of these include XAxis, YAxis, Legend, Tooltip, Brush, and CartesianGrid, which are mostly self explanatory. Brushes are used to allow the user to filter their data into smaller sections to make observing the data easier, especially in larger data sets.\\

\noindent The format for passing data to each of these components is mostly the same, although the actual content of the data changes by index. In general, the data being passed is of the following format.

\begin{javascriptcode}
{
  graph1: [
            {
              name: 'Left Channel',
              ndsi: results.ndsiL,
              biophony: results.biophonyL,
              anthrophony: results.anthrophonyL
            },
            {
              name: 'Right Channel',
              ndsi: results.ndsiR,
              biophony: results.biophonyR,
              anthrophony: results.anthrophonyR
            }
          ],
  graph2: [
            {
              name: 'NDSI',
              leftChannel: results.ndsiL,
              rightChannel: results.ndsiR
            },
            {
              name: 'Biophony',
              leftChannel: results.biophonyL,
              rightChannel: results.biophonyR
            },
            {
              name: 'Anthrophony',
              leftChannel: results.anthrophonyL,
              rightChannel: results.anthrophonyR
            }
          ],
  graph3: []
}
\end{javascriptcode}

\noindent Some indices include more graphs naturally, as to better visualize the different data outputs. The code above represents the simplest graph, that being the NDSI bar charts. These three graphs are lists of JSON objects, inside of a parent JSON object, that is in turn passed into the NDSI component responsible for creating the graphs. Graph1 represents the two bar group visualization, while Graph2 represents the three bar group one. Note that Graph3 is used only when the dataset is comprised of multiple files, and is used for calculating an overall NDSI over all the files in the dataset. The name field is used for determing which data the X axis should track, while the other fields represent actual data to be represented by the graphs in the visualization component, which are passed as data keys in the components.

\subsubsection{Line Charts}

\begin{htmlcode}
<LineChart width={900} height={600} data={data} >
  <CartesianGrid strokeDasharray="3 3"/>
  <XAxis dataKey="name" label={xLabel}/>
  <YAxis label={yLabel}/>
  <Legend />
  <Tooltip/>
  <Line type='natural' dataKey={firstDataKey}
                      stroke='#8884d8'
                      dot={false} />
  <Line type='natural' dataKey={secondDataKey}
                      stroke='#82ca9d'
                      dot={false} />
  <Brush />
</ LineChart>
\end{htmlcode}

\noindent The code above is used for creating the line charts seen in the NDSI, ACI, and Bioacoustic index, as well as the line charts for evaluating outliers. This flexibility is allowed as we can pass the label names and data key parameters through the components depending on which index was chosen.

\begin{htmlcode}
<ReferenceLine y={adiLeft} label="ADI Left" stroke="#433eaf"/>
\end{htmlcode}

\noindent For the ADI and AEI indices, a reference line is included as well, which is elaborated on more in the Data Visualization Research section of this paper.

\subsubsection{Bar Graphs}

\begin{htmlcode}
<BarChart width={900} height={600} data={graph1}>
  <CartesianGrid strokeDasharray="3 3" />
  <XAxis dataKey="name" label="Channel"/>
  <YAxis label="Value"/>
  <Tooltip />
  <Legend />
  <Bar dataKey="ndsi" fill="#8884d8" />
  <Bar dataKey="biophony" fill="#82ca9d" />
  <Bar dataKey="anthrophony" fill="#e79797" />
</ BarChart>
\end{htmlcode}

The NDSI bar charts are a bit more complicated to make flexible. One of the NDSI charts includes three different bar groups, while the other only includes two. So, making one all encompassing React component is out of the question. Thus, the code above is used for the three bar group visualization, with similar code used for the two bar graph visualization.

\subsubsection{Area Graphs}

\begin{htmlcode}
<AreaChart width={900} height={600} data={data} >
  <CartesianGrid strokeDasharray="3 3"/>
  <XAxis dataKey="name" label={xLabel}/>
  <Legend />
  <YAxis label={yLabel}/>
  <Tooltip/>
  <Area type='monotone' dataKey={firstDataKey}
                        stackId="1"
                        stroke='#8884d8'
                        fill='#8884d8' />
  <Brush />
</ AreaChart>
\end{htmlcode}

The area graphs are also very flexible, as they are not used as much in this service. Thus, using the code above, we can easily include area graphs wherever they are needed regardless of the index (which is only Bioacoustic index) by again passing parameters through components.

\subsubsection{Composed Charts}

\begin{htmlcode}
<ComposedChart width={900} height={600} data={data}>
  <CartesianGrid strokeDasharray="3 3" />
  <XAxis dataKey="name" label={xLabel}/>
  <YAxis label={yLabel}/>
  <Tooltip />
  <Legend />
  <Brush dataKey='name' height={30} />
  <Bar dataKey={firstDataKey} fill="#8884d8" />
  <Bar dataKey={secondDataKey} fill="#82ca9d" />
  <Line type='monotone' dataKey={firstDataKey}
                        stroke='#8884d8'
                        dot={false}/>
  <Line type='monotone' dataKey={secondDataKey}
                        stroke='#82ca9d'
                        dot={false}/>
</ ComposedChart>
\end{htmlcode}

\noindent This component is used only once, in the NDSI comparison charts. A composed chart includes more than one data representation, this one including bar graphs \textit{and} line graphs. The purpose of this is explained in the Data Visualization Research section of this paper. The code for doing this is mostly the same as the respective bar graph and line graph components, just used together in a parent ComposedChart component provided by Recharts.

