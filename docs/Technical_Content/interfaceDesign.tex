\subsection{User Interface Design}
When designing an interface, the purpose of the service comes into question. It is important to define what drives each page of the application, and what the core functionality of it is in order to choose the best layout and color scheme for the component\textquotesingle s representation and the user\textquotesingle s accessibility. In addition, identifying the core user base is also important in designing the interface.

\subsubsection{Interface Color Selection}
For this service, some pages are visually driven, with little text on the screen, while some are text driven. The Catalog page for instance relies on the data visualization component as the main focal point, while the settings page contains only text. This presents a bit of a conflict as to which colors to choose for a service wide color pallete.\par
With pages that are mainly visually driven, like the Catalog page, it is best to use darker background colors to help the visuals stand out compared to the rest of the page. For text however it is best to use lighter color backgrounds as to not cause strain on the reader\textquotesingle eyes. Thus it is decided that the service will be mostly light colored backgrounds, with some colored components.\par
For the Catalog page, it is important for the data visualizations to stand out. Contrast in pages is important for drawing attention to elements and directing the user\textquotesingle s eyes. The analysis section of the page should have a background color as to make it stand out against the rest of the page.\par
As for the actual color pallete selection, our sponsor has a logo made up for his research that contains core colors of yellow and black. We liked this logo from the get go and wanted to incorporate these colors into the application. We aim to follow the 60-30-10 rule using a pallete of white yellow and black.

\subsubsection{Page Organization}
Page organization is important for putting elements in a logical and meaningful place. Placing related content together helps the end user be more efficient, especiall on pages like the Job Creation page. The following thought process went into each page.\par
The Catalog page contains the user\textquotesingle s results from past jobs, and allows them to filter through them and view the analysis for whichever they\textquotesingle d like. The main component of this page is again the data visualizations, so it is important that this component take up most of the space on the page. The side components, that being the filtering section and the results table, takes up a third of the screen and are aligned to the left, while the data visualizations take up the other two thirds and the right side of the screen. This helps to divide up the content for the user, and make filtering and navigating the results table easier.\par
KEITH SECTION ON SETTINGS PAGE\par
BRITA SECTION ON NEW JOBS PAGE

\subsubsection{Frontend Frameworks}
A big part of getting the front end of the service up and running as quickly as we did was the use of frameworks. The core frameworks for this project are explained in the Infrastructure section of this project, so this will only cover the core frontend frameworks. The core frameworks used are as follows:
\begin{itemize}
  \item Material-UI
  \item Bootstrap
  \item D3
\end{itemize}
Material-UI is a framework for React that includes new components to easily add into the existing pages, handling styling for these components to help save time in development. This framework is used heavily in both the Catalog and Job Creation pages.\par
Bootstrap is a CSS framework made to handle screen size changes without hassle on the developer\textquotesingle s end. Bootstrap divides the page up into twelve parts, containing rows and columns. Components can then be placed into their respective rows and columns, including nesting, to create a functional and dynamic interface. This framework is especially useful for us to handle changing screen sizes.\par
For the analysis part of the Catalog page, we needed a framework to turn our JSON data into graphics based on the user selected result from the result table. D3 is a framework for doing just that, and is in charge of creating the respective data visualizations for each index.
