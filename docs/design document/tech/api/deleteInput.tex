\subsubsection{Delete Input}
Delete an existing input by \codesnip{inputId}.

\paragraph{DELETE} \mbox{}\\[\codeheaderspace]
\begin{htmlcode}
https://<base-url>/inputs/:inputId
\end{htmlcode}

\paragraph{URL Parameters} \mbox{}\\[\longtableheaderspace]
\begingroup
\renewcommand{\arraystretch}{\cellpaddingvertical}
\begin{longtable}{| m{\fieldcolwidth} | m{\typecolwidth} | m{\desccolwidthlg} |}
  \hline
  \tablehead{Field}
  & \tablehead{Type}
  & \tablehead{Description}
  \\ \hline

  \codesnip{inputId}
  & String
  & The \codesnip{inputId} of the input to be deleted.
  \\ \hline
\end{longtable}
\endgroup

\paragraph{Example Request: DELETE} \mbox{}\\[\codeheaderspace]
\begin{htmlcode}
https://<base-url>/inputs/input-1-uuid
\end{htmlcode}

\paragraph{Response Codes} \mbox{}\\[\responseheaderspace]
\begingroup
\renewcommand{\arraystretch}{\cellpaddingvertical}
\begin{longtable}{| m{\rescodecol} | m{\resconditioncol} |}
  \hline
  \tablehead{Code}
  & \tablehead{Response}
  \\ \hline

  \hspace{2.5mm} 200
  & The input with the specified \codesnip{inputId} was deleted if it existed. Otherwise, nothing happened.
  \\ \hline
\end{longtable}
\endgroup

\paragraph{Response Body Fields} \mbox{}\\[\longtableheaderspace]
\begingroup
\renewcommand{\arraystretch}{\cellpaddingvertical}
\begin{longtable}{| m{\fieldcolwidth} | m{\typecolwidth} | m{\desccolwidthlg} |}
  \hline
  \tablehead{Field}
  & \tablehead{Type}
  & \tablehead{Description}
  \\ \hline

  \codesnip{success}
  & Boolean
  & A confirmation of whether or not the specified input was removed.
  \\ \hline

  \codesnip{message}
  & String
  & A message containing information regarding the deletion.
  \\ \hline

  \codesnip{jobs}
  & Object[]
  & A list of \codesnip{jobId}s of all Jobs associated with the input that was
  deleted. These Jobs have also been deleted in the process.
  \\ \hline
\end{longtable}
\endgroup

\paragraph{Example Response Body} \mbox{}\\[\codeheaderspace]
\begin{jsoncode}
{
  "success": true,
  "message": "Successfully deleted input with inputId: input-1-uuid.",
  "jobs": [
    "job-1a-uuid"
  ]
}
\end{jsoncode}
