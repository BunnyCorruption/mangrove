\subsubsection{Get All Specifications}
Retrieve the details of all existing specifications.

\paragraph{GET} \mbox{}\\[\codeheaderspace]
\begin{htmlcode}
https://<base-url>/specs
\end{htmlcode}

% TODO: Implement pagination
%\paragraph{URL Parameters} \mbox{}\\[\longtableheaderspace]
%\begingroup
%\renewcommand{\arraystretch}{\cellpaddingvertical}
%\begin{longtable}{| m{\fieldcolwidth} | m{\typecolwidth} | m{\desccolwidthlg} |}
%  \hline
%  \tablehead{Field}
%  & \tablehead{Type}
%  & \tablehead{Description}
%  \\ \hline
%
%  \codesnip{page}
%  & Number
%  &
%  \\ \hline
%
%  \codesnip{perPage}
%  & Number
%  &
%  \\ \hline
%\end{longtable}
%\endgroup

\paragraph{Response Body Fields} \mbox{}\\[\longtableheaderspace]
\begingroup
\renewcommand{\arraystretch}{\cellpaddingvertical}
\begin{longtable}{| m{\fieldcolwidth} | m{\typecolwidth} | m{\desccolwidthlg} |}
  \hline
  \tablehead{Field}
  & \tablehead{Type}
  & \tablehead{Description}
  \\ \hline

  % TODO: Implement pagination.
  %\codesnip{pages}
  %& Number
  %& The total number of pages.
  %\\ \hline

  \codesnip{count}
  & Number
  & The total number of specs.
  \\ \hline

  \codesnip{specs}
  & Object[]
  & A list of all specs. See the Get Spec request for a complete listing of properties. % TODO: Implement pagination.
  \\ \hline
\end{longtable}
\endgroup

\paragraph{Example Response Body} \mbox{}\\[\codeheaderspace]
\begin{jsoncode}
{
  "count": 1,
  "specs": [
    {
      "specId": "spec-a-uuid",
      "metric": "aci",
      "minFreq": 0,
      "maxFreq": 16000,
      "j": 30,
      "fftW": 10
    }
  ]
}
\end{jsoncode}
