\subsubsection{Create Specification}
Create a specification for a given metric, including the Acoustic Complexity Index (ACI), Acoustic Diversity Index (ADI), Acoustic Evenness Index (AEI), Bioacoustic Index (BI), Normalized Difference Soundscape Index (NDSI), and Root Mean Square (RMS).

\paragraph{PUT} \mbox{}\\[\codeheaderspace]
\begin{htmlcode}
https://<base-url>/specs
\end{htmlcode}

\paragraph{Header} \mbox{}\\[\longtableheaderspace]
\begingroup
\renewcommand{\arraystretch}{\cellpaddingvertical}
\begin{longtable}{| m{\fieldcolwidth} | m{\typecolwidth} | m{\desccolwidthlg} |}
  \hline
  \tablehead{Field}
  & \tablehead{Type}
  & \tablehead{Description}
  \\ \hline

  \codesnip{Content-Type}
  & String
  & \codesnip{"application/json"}
  \\ \hline
\end{longtable}
\endgroup

\paragraph{Request Body Fields} \mbox{}\\[\longtableheaderspace]
\begingroup
\renewcommand{\arraystretch}{\cellpaddingvertical}
\begin{longtable}{| m{\fieldcolwidth} | m{\typecolwidth} | m{\metriccolwidth} | m{\desccolwidthsm} |}
  \hline
  \tablehead{Field}
  & \tablehead{Type}
  & \tablehead{Metric}
  & \tablehead{Description}
  \\ \hline

  \codesnip{metric}
  & String
  &
  & The metric to be calculated for a given spec. Possible values: \codesnip{"aci"}, \codesnip{"adi"}, \codesnip{"aei"}, \codesnip{"bi"}, \codesnip{"ndsi"}, \codesnip{"rms"}.
  \\ \hline

  \codesnip{minFreq}
  & Number
  & ACI, BI
  & The minimum frequency to use when calculating the value, in Hertz.
  \\ \hline

  \codesnip{maxFreq}
  & Number
  & ACI, ADI, AEI, BI
  & The maximum frequency to use when calculating the value, in Hertz.
  \\ \hline

  \codesnip{j}
  & Number
  & ACI
  & The cluster size, in seconds.
  \\ \hline

  \codesnip{fftW}
  & Number
  & ACI, BI, NDSI
  & The fast Fourier transform window.
  \\ \hline

  \codesnip{dbThreshold}
  & Number
  & ADI, AEI
  & The threshold.
  \\ \hline

  \codesnip{freqStep}
  & Number
  & ADI, AEI
  & The size of frequency bands.
  \\ \hline

  \codesnip{shannon}
  & Boolean
  & ADI
  & Set to \codesnip{true} to use the Shannon\textquotesingle s diversity index.
  \\ \hline

  \codesnip{anthroMin}
  & Number
  & NDSI
  & The minimum value of the range of frequencies of the anthrophony.
  \\ \hline

  \codesnip{anthroMax}
  & Number
  & NDSI
  & The maximum value of the range of frequencies of the anthrophony.
  \\ \hline

  \codesnip{bioMin}
  & Number
  & NDSI
  & The minimum value of the range of frequencies of the biophony.
  \\ \hline

  \codesnip{bioMax}
  & Number
  & NDSI
  & The maximum value of the range of frequencies of the biophony.
  \\ \hline

  % TODO: Add RMS metric parameters.
\end{longtable}
\endgroup

\paragraph{Example Request Body} \mbox{}\\[\codeheaderspace]
\begin{jsoncode}
{
  "metric": "aci",
  "minFreq": 0,
  "maxFreq": 16000,
  "j": 30,
  "fftW": 10
}
\end{jsoncode}

\paragraph{Response Codes} \mbox{}\\[\responseheaderspace]
\begingroup
\renewcommand{\arraystretch}{\cellpaddingvertical}
\begin{longtable}{| m{\rescodecol} | m{\resconditioncol} |}
  \hline
  \tablehead{Code}
  & \tablehead{Response}
  \\ \hline

  \hspace{2.5mm} 201
  & A new specification is created and returned.
  \\ \hline

  \hspace{2.5mm} 200
  & No new specification is created. The fields provided in the request match an existing specification, which is returned.
  \\ \hline
\end{longtable}
\endgroup

\paragraph{Response Body Fields} \mbox{}\\[\longtableheaderspace]
\begingroup
\renewcommand{\arraystretch}{\cellpaddingvertical}
\begin{longtable}{| m{\fieldcolwidth} | m{\typecolwidth} | m{\metriccolwidth} | m{\desccolwidthsm} |}
  \hline
  \tablehead{Field}
  & \tablehead{Type}
  & \tablehead{Metric}
  & \tablehead{Description}
  \\ \hline

  \codesnip{specId}
  & String
  &
  & A unique identifier for the specification.
  \\ \hline

  \codesnip{metric}
  & String
  &
  & The metric to be calculated for a given spec. Possible values: \codesnip{"aci"}, \codesnip{"adi"}, \codesnip{"aei"}, \codesnip{"bi"}, \codesnip{"ndsi"}, \codesnip{"rms"}.
  \\ \hline

  \codesnip{minFreq}
  & Number
  & ACI, BI
  & The minimum frequency to use when calculating the value, in Hertz.
  \\ \hline

  \codesnip{maxFreq}
  & Number
  & ACI, ADI, AEI, BI
  & The maximum frequency to use when calculating the value, in Hertz.
  \\ \hline

  \codesnip{j}
  & Number
  & ACI
  & The cluster size, in seconds.
  \\ \hline

  \codesnip{fftW}
  & Number
  & ACI, BI, NDSI
  & The fast Fourier transform window.
  \\ \hline

  \codesnip{dbThreshold}
  & Number
  & ADI, AEI
  & The threshold.
  \\ \hline

  \codesnip{freqStep}
  & Number
  & ADI, AEI
  & The size of frequency bands.
  \\ \hline

  \codesnip{shannon}
  & Boolean
  & ADI
  & Set to \codesnip{true} to use the Shannon\textquotesingle s diversity index.
  \\ \hline

  \codesnip{anthroMin}
  & Number
  & NDSI
  & The minimum value of the range of frequencies of the anthrophony.
  \\ \hline

  \codesnip{anthroMax}
  & Number
  & NDSI
  & The maximum value of the range of frequencies of the anthrophony.
  \\ \hline

  \codesnip{bioMin}
  & Number
  & NDSI
  & The minimum value of the range of frequencies of the biophony.
  \\ \hline

  \codesnip{bioMax}
  & Number
  & NDSI
  & The maximum value of the range of frequencies of the biophony.
  \\ \hline

  % TODO: Add RMS metric parameters.
\end{longtable}
\endgroup

\paragraph{Example Response Body} \mbox{}\\[\codeheaderspace]
\begin{jsoncode}
{
  "specId": "spec-a-uuid",
  "metric": "aci",
  "minFreq": 0,
  "maxFreq": 16000,
  "j": 30,
  "fftW": 10
}
\end{jsoncode}

\paragraph{Error Handling} \mbox{}\\[\longtableheaderspace]
\begingroup
\renewcommand{\arraystretch}{\cellpaddingvertical}
\begin{longtable}{| m{\errconditioncol} | m{\errcodecol} | m{\errbodycol} |}
  \hline
  \tablehead{Condition}
  & \multicolumn{2}{|l|}{\tablehead{Response}}
  \\ \hline

  A field was included that was not listed as a valid request body field.
  & 400
  & An object containing a single field, \codesnip{message} (String), identifying the invalid field from the request. This may result from a valid request body field being included for the wrong \codesnip{metric}.
  \\ \hline

  A field contains an invalid value.
  & 400
  & An object containing a single field, \codesnip{message} (String), identifying the field in question and its possible values.
  \\ \hline
\end{longtable}
\endgroup
