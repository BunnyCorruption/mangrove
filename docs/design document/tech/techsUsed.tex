\subsection{Technologies Used}

\paragraph{Front End} \mbox{} \\
\textit{Technologies used: Electron, ReactJS, Recharts}\par
React is a JavaScript library made for creating user interface components in an easily reusable way. Each React component allows for an HTML mockup with variables that are used as placeholders that we can pass info to. Each component can have parameters, called props, and states, which help keep track of changes in the data flow. This is useful because we can create HTML templates that are useable with different parameters, for quickly making new and similar web pages.\par
React uses a component hierarchy that allows for component props and states to be passed down along to subcomponents, allowing for a responsive and modular page design. An example in this application includes the Results Catalog page. The table of results available to the user can be filtered based on user defined name, completed or in progress jobs, and the index used in the results. With React, these options exist as states, and when they change, their respective handlers update the component props, which are then passed down the hierarchy of components to create a filtered results table based on the user\textquotesingle s input. In addition to the results table, the analysis view also relies on input from other components, namely the results table, to determine which information to display. When the user selects a result from the results table, the state again changes to supply the correct props to the analysis component, which in turn renders the data visualizations given the selected results.\par
Electron is a tool for creating desktop applications using JavaScript, HTML, and CSS. Electron was originally created to make the Atom IDE but was adopted by other companies to make desktop applications like Spotify. Electron also includes cross platform support as well as crash reporting. The cross platform abilities are crucial for this project because the sponsor uses a Mac device, but even further, cross platform support is all around a good idea.\par
Recharts is the library used for creating the data visualizations from job results. It is built off the D3 library and offers a great API for creating easy to read, high quality graphs using React components. In addition, it offers a wide variety of graph types which is essential to this project as each index being used requires different visual representation. Recharts also allows for interactivity with the graphs themselves, so allowing the user to listen to audio files with the click of a data point is possible. Overall, Recharts seems the best entry level graphing library for this project.\par

\paragraph{Local Back End} \mbox{}\\
\textit{Technologies used: Express, NodeJS, MongoDB, Mongoose, JavaScript, Redis, R}\par
Express is a JavaScript library for creating a sort of \textquotesingle skeleton\textquotesingle\ of a website. By using Express, we can create an outline of our site\textquotesingle s backend local infrastructure. This is useful for the development of the API we will be using to allow all the working pieces of the project to interact with each other.\par
Using Node and MongoDB along with Mongoose, we can interface and query the Mongo database, hereon refered to as the local database. The reason that MongoDB is the best choice for our particular project is that Mongo allows for more dynamic database objects than a SQL database. An example of this would be in the output of the algorithms being run against the sound files. Some of the output is done in lists, and depending on the size of the input files, the number of those lists is arbitrary. This characteristic makes using a SQL database very difficult because we\textquotesingle d need an arbitrary number of rows for each list that is output from the processing. With MongoDB, we can include lists as a field in our database, and there these lists can be populated in a much more compact way. In addition, during planning, we came up with an efficient way to map parameter research, which we are referring to as jobs, to the inputs they are run on. MongoDB allows reference IDs as fields, so we are going to be implenting a M to N relationship, where a set of inputs, or multiple sets, will be mapped to many different jobs. This will allow the user a lot of freedom when choosing what kind of analysis they want to run on their data.\par
Our service\textquotesingle s backend API is written in JavaScript, allowing for JSON objects to be passed back to the client for processing in the various pages. Further API documentation can be found in the Application Programming Interface section of this report.\par
As for the processing of jobs on indices we used Redis. This cache server allows us to use a package by the name of Bull. The combination of Bull and Redis allow the server to run jobs outside of the Javascript server's event loop. This means that we can process multiple jobs at the same time all the while keeping the server open to new requests. Without Redis our server would be "busy" during any processing and wouldn't be able to communicate with the client.
