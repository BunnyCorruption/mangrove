\subsubsection{Soundecology Package Dependencies}
The core \codesnip{soundecology} package does not handle wav file conversion to Wav object that is needed for use in the algorithms. This is handled by a package named \codesnip{tuneR}, which takes in audio files and converts them to R objects compatible with the R algorithms in \codesnip{soundecology}. This is the only dependency that is of any note, as it will require its own set of instructions during our processing of the user\textquotesingle s files. The script for converting a user file into a Wav object is the following

\begin{javascriptcode}
tdir <- getwd()
tfile <- file.path(tdir, "SoundFileName.wav")
newWobj <- readWave(tfile)
result <- ndsi(newWobj)
\end{javascriptcode}

In order for this to work, the current working directory must be set to where the user\textquotesingle s files are located, and the SoundFileName must match the user\textquotesingle s as well. Luckily \codesnip{soundecology} \textit{does} have a method, \verb|multiple_sounds|, for going through all files in a subdirectory, and handles this natively. However \verb|multiple_sounds| only goes through a directory, and cannot be used for selecting arbitrarily specified files.\\

\begin{javascriptcode}
newWobj <- readWave(tfile, from=1, to=60, units="seconds")
\end{javascriptcode}

Using the code above, we can split up large sound files to be more efficient for processing. This code in particular splices the \verb|tfile| object into the first minute. More information on this is seen in the Benchmarking section.
