\subsection{Broader Impacts}
Our project will help analysis in the field of soundscape ecology, which is currently a new and niche research field. This field studies the relationship between human, environmental and biological sounds in nature and urban environments. Researchers need better tools for analysis to understand the effect human behavior is having on various natural environments and biological populations, such as noise pollution interfering with an organism\textquotesingle s ability to detect predators.\par
We would like to produce tools to promote collaboration between researchers and research teams. This will provide more relevant data for study and draw more diverse voices into the field, with an overall goal of advancing the aims of biological conservation in a complicated world. From meeting with our sponsor, it\textquotesingle s been explained that there is tremendous interest outside of the field in the analysis and research being done here. The only problem is that there is yet to be a cost efficient way for researchers to analyze and visualize their data for reports and presentations. Having a public tool set for researchers to use in analyzing soundscape data would be highly beneficial to the field and those interested in it. Even further, stretch goals for the project and possible future versions include machine learning algorithms that will identify both animal and human made sound.\par
Whether it be the identification of birds in a local park or the types of airplanes that fly over, the ability to identify these sounds can help researchers discover the fauna of large ecosystems and the effects of man made sound on them. Overall, creating software that is one of a kind to spearhead a whole research field is an incredibly unique opportunity that the team is looking forward to tackling. Hopefully with this tool, the field of soundscape ecology will become less abstracted and open up to more teams around the world.
