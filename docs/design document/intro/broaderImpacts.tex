\subsection{Broader Impacts}
Though it is relatively young, only having been around since the 1970\textquotesingle s, there is increasing interest from outside the field of soundscape ecology, looking into the analysis and research being performed within. The biggest problem facing the field is that there is yet to be a cost efficient way for researchers to analyze and visualize their data. Most commonly, soundscape ecologists turn to professional ornithologists to analyze sound files, and this can be costly. Without affordable and reliable ways of analyzing raw sound data, these researchers are limited in their ability to progress the field via research publication. Having a public tool set for researchers to use in analyzing soundscape data has the ability to directly benefit the field and those interested in its findings.