\subsection{Project Progress and Successes}
\subsubsection{Tools for Success}
Our meeting schedule changed throughout the two semesters based on the needs of the project. During the initial design process, we met at least weekly along with daily Discord discussions. These weekly meetings lasted until we all felt we had made a considerable amount of progress on the project. Once most of the design decisions had been made and we decided to start working on the application, we met on an as needed basis. At this point, Discord communication became more frequent. We created Discord channels related to certain areas of the project, like backend, frontend and a channel to discuss ideas and plans for writing this design document. Documents related to the design process are stored in a shared Google Drive folder. Github has been used for our frontend and backend code, as well as our \LaTeX{} design document.\par
Our team manager has kept up with setting topics and goals before each meeting, as well as keeping track of ideas discussed in meetings. These are also kept in our Google Drive folder. This practice also allows us to keep track of our progress and evaluate if we are behind schedule on any part of the project.\par
We have kept values in mind during design and development. This includes making sure we are considering the needs of our sponsor and future users.\par
The concept of creating 100 ideas has been done throughout our team meetings and Discord communication, with lengthy discussions becoming commonplace in the early stages of this project. This is especially true when thinking about design decisions, many ideas are suggested in our meetings until the team has agreed that the best ones have been found. We have dealt with risks in a similar way to the 100 ideas method. By trying to think ahead to potential risks that may arise in the future, we were able to plan around them or think of a new idea so that the risk could be avoided.\par
Delegating tasks has been a very useful practice throughout our project. This has aided in the design process by splitting up sections of the design document across our team. New ideas were discovered and discussed in meeting and by doing this we have been able to cover more material. In development this has also been useful, as our team is split up into frontend and backend with subroles within each group, ensuring that no aspect of the project is forgotten.\par

\subsubsection{Milestone Progress}
\paragraph{Phase 0 - Requirement Gathering and Initial Design} \mbox{}\\[\paragraphheaderspace]
The first milestone in this section, gathering requirements has been completed throughout various meetings with our team, sponsor, TAs and professor. Discussions with our sponsor have clarified what it means for our project to be successful and determined which features should be considered stretch goals. In meeting with our professor, the importance of attempting to implement stretch goal features has been discussed. Our team has agreed on pursuing machine learning  and collaboration stretch goals and we have allocated time for this as shown in the Milestones section of this paper.\par
The aspects of project design, including the database, API/backend and frontend have been through various iterations and continued to improve throughout the project. A more detailed account of this process can be found in the Design Iterations section of this document.\par
\paragraph{Phase 1 - Research and Prototyping} \mbox{}\\[\paragraphheaderspace]
We made a good amount of progress prototyping our server and client applications compared to the milestone dates set at the start of semester one. Progress on the server side application included some API requests. The backend team continued to develop the requests outlined in the Application Programming Interface section.\par
Prototyping of the client application was also off to a good start. The application runs on Electron in a development environment and the front end team made progress on pages of the application. The client prototype application had a navigation panel for each of the pages that have started to be implemented, job creation, job queue and catalog.\par
Progress on the catalog page included job filtering and searching. On this page, progress on Recharts visualizations for results had also been made for all indices and data structured the way it will be in the database could be viewed in Recharts graphs. By the end of December, these two features were working together by showing job results of any sample job searched, as well as job comparisons. The job queue page, where a user starts new jobs, included UI selection of indices and parameters. The next step on that page was file input selection for jobs and after that was implemented, requests could be sent to the API to make a new job.\par
\newpage
\paragraph{Implementation and Stretch Goals} \mbox{}\\[\paragraphheaderspace]
Implementation of the local desktop application began in early January. After all aspects of the client application were functioning with sample data, we started to include API requests to get both applications working with one another. The backend team continued to create more of the necessary requests for our application while the front end team worked with sample data. We were successful in meeting the date set in the milestones section, January 25, 2018, for the implementation of the desktop application. Some backend testing had also already taken place at that time.\par
Implementation of the AWS backend application may be the only section that we underestimated the time needed to complete in our milestones. Because of this it was decided to move this portion of the project into stretch goals.\par
Stretch goals were not been our primary concern, in order to get all required goals completed first. We have had one designated member of our team researching possible implementations of our stretch goals through machine learning. When the required features were completed, the rest of the team would be able to move their focus to this area and the research already done would serve as a good starting point. However, due to increased time spent working on more advanced features in the frontend, this work stayed with just one group member.\par

\paragraph{Sponsor Feedback and Additional Features} \mbox{}\\[\paragraphheaderspace]
After the API was fully implemented, files were able to be processed and the client application could use real data given by the results of running jobs. After all major features were complete, Mangrove was set up on our sponsor\textquotesingle s computer and we were given feedback and suggestions for additonal features on both the client and server applications. We shifted our focus to implementing this feedback, as well as correcting any bugs found.\par
Progress made on the client application during the second semester include researching and implementing the most useful visualizations for results of each type of job and adding the ability to playback audio when a data point is clicked. Automating the collection of metadata from the names of sound files was added to improve the user\textquotesingle s experience by the request of our sponsor. Other client features completed in the second semester include exporting results to CSV files and live updates of graphs as jobs are processed. The login page was also completed, along with user authentication and persistent user sessions.\par
