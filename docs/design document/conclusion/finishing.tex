\subsection{Final Remarks}
Some of the major successes that come with Mangrove include the fact that this piece of software really is one of a kind. There is no known software available solely for soundscape ecology researchers to both organize and process their datasets. One of the features that makes this possible would be the graphs that come along with Mangrove. Finally, researchers have access to visuals that describe the numbers they are currently forced to work with. The vision is that in using these graphs, researchers can now publish their findings in a consumable fashion for both other researchers and the general public. Another great feature of Mangrove is the processing capabilities. Before, researchers like our sponsor would have to work in the terminal to run R scripts that had the potential to fail and nullify any progress made. Now, each file is run with live updates to show the user that work is actually happening. Just because one file may have failed doesn\textquotesingle t mean that the entire operation is now botched. Providing an intuitive interface for researchers to use these algorithms to analyze their sound files is a valuable resource in this blooming field.\par
Aside from the technical achievements of Mangrove, through development we have also brought to light some challenges and discrepencies in the field of sound ecology. We have been able to identify problems in the algorithms being used by researchers and are able to propose changes (and have made changes even) to the algorithms to improve validity of data. Additionally, our machine learning research has shed a great deal of light on the possibilities of machine listening and the implications of that practice as it pertains to soundscape ecology research.\par
Moving forward, we hope that the collaborative abilities planned by our current design are brought to fruition. The ultimate goal of Mangrove is to be open source and public facing in order to promote open data in the realm of soundscape ecology. We also hope that the research done regarding the index algorithms and their validity and purpose in the field is found useful to researchers, as we feel there is a great deal of discrepency as to the actual usefulness of these algorithms. Finally, we hope that machine learning is integrated into Mangrove in the future using the research and work done in our two semesters working with it.\par
Overall, Mangrove has been a great success. Crunch time was minimal, and the requirements we set out to reach have been met. We feel that Mangrove has provided everything our sponsor wanted and then some, and we know that both him and our team are proud of the work done on Mangrove.\par
