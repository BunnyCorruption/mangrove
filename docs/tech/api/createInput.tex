\subsubsection{Create Input}
Create an input based on a WAV audio file and a set of fields that describe the conditions in which the file was recorded.

\paragraph{PUT} \mbox{}\\[\codeheaderspace]
\begin{htmlcode}
https://<base-url>/inputs
\end{htmlcode}

\paragraph{Header} \mbox{}\\[\longtableheaderspace]
\begingroup
\renewcommand{\arraystretch}{\cellpaddingvertical}
\begin{longtable}{| m{\fieldcolwidth} | m{\typecolwidth} | m{\desccolwidthlg} |}
  \hline
  \tablehead{Field}
  & \tablehead{Type}
  & \tablehead{Description}
  \\ \hline

  \codesnip{Content-Type}
  & String
  & \codesnip{"multipart/form-data"}
  \\ \hline
\end{longtable}
\endgroup

\paragraph{Request Body Fields} \mbox{}\\[\longtableheaderspace]
\begingroup
\renewcommand{\arraystretch}{\cellpaddingvertical}
\begin{longtable}{| m{\fieldcolwidth} | m{\typecolwidth} | m{\desccolwidthlg} |}
  \hline
  \tablehead{Field}
  & \tablehead{Type}
  & \tablehead{Description}
  \\ \hline

  \codesnip{json}
  & Text
  & A description of the input being created. This should be a stringified JSON object containing the fields below.
  \\ \hline

  \hspace{3mm} \codesnip{site}
  & String
  & The name of the site in which the input was recorded.
  \\ \hline

  \hspace{3mm} \codesnip{series}
  & String
  & The name of the series of recordings that the input is a part of.
  \\ \hline

  \hspace{3mm} \codesnip{recordTimeMs}
  & Number
  & The time at which the input audio file was recorded, listed in milliseconds since the Unix epoch.
  \\ \hline

  \hspace{3mm} \codesnip{coords}
  & Object
  & The coordinates of the location in which the input was recorded.
  \\ \hline

  \hspace{6mm} \codesnip{lat}
  & Number
  & The latitude of the recording location.
  \\ \hline

  \hspace{6mm} \codesnip{long}
  & Number
  & The longitude of the recording location.
  \\ \hline

  \codesnip{file}
  & File
  & The WAV file to be analyzed in any given job that references this input. Note: Be sure that this is sent last in the request.
  \\ \hline
\end{longtable}
\endgroup

\paragraph{Example Request Body} \mbox{}\\[\longtableheaderspace]
\begingroup
\renewcommand{\arraystretch}{\cellpaddingvertical}
\begin{longtable}{| m{\fieldcolwidth} | m{\valcolwidth} |}
  \hline
  \tablehead{Field}
  & \tablehead{Value}
  \\ \hline

  \codesnip{json}
  & \begin{mdframed}[backgroundcolor=bgcolor,skipabove=0pt,skipbelow=0pt,topline=false,bottomline=false,rightline=false,leftline=false]
  \begin{minted}[gobble=2]{json}
  {
    "site": "UCF Arboretum",
    "series": "Hurricane Irma",
    "recordTimeMs": 1505016000000,
    "coords": {
      "lat": 28.596238,
      "long": -81.191381
    }
  }
  \end{minted}
  \end{mdframed}
  \\ \hline

  \codesnip{file}
  & \codesnip{irma-1.wav}
  \\ \hline
\end{longtable}
\endgroup

\paragraph{Response Codes} \mbox{}\\[\responseheaderspace]
\begingroup
\renewcommand{\arraystretch}{\cellpaddingvertical}
\begin{longtable}{| m{\rescodecol} | m{\resconditioncol} |}
  \hline
  \tablehead{Code}
  & \tablehead{Response}
  \\ \hline

  \hspace{2.5mm} 201
  & A new input is created and returned.
  \\ \hline

  \hspace{2.5mm} 200
  & No new input is created. The fields provided in the request match an existing input, which is returned.
  \\ \hline
\end{longtable}
\endgroup

\paragraph{Response Body Fields} \mbox{}\\[\longtableheaderspace]
\begingroup
\renewcommand{\arraystretch}{\cellpaddingvertical}
\begin{longtable}{| m{\fieldcolwidth} | m{\typecolwidth} | m{\desccolwidthlg} |}
  \hline
  \tablehead{Field}
  & \tablehead{Type}
  & \tablehead{Description}
  \\ \hline

  \codesnip{inputId}
  & String
  & A unique identifier for the audio input file.
  \\ \hline

  \codesnip{path}
  & String
  & The path to the input file on the host machine, from the root input folder. Note: This will always be a path to a WAV file.
  \\ \hline

  \codesnip{site}
  & String
  & The name of the site in which the input was recorded.
  \\ \hline

  \codesnip{series}
  & String
  & The name of the series of recordings that the input is a part of.
  \\ \hline

  \codesnip{recordTimeMs}
  & Object
  & The time at which the input audio file was recorded, listed in milliseconds since the Unix epoch.
  \\ \hline

  \codesnip{coords}
  & Object
  & The coordinates of the location in which the input was recorded.
  \\ \hline

  \hspace{3mm} \codesnip{lat}
  & Number
  & The latitude of the recording location.
  \\ \hline

  \hspace{3mm} \codesnip{long}
  & Number
  & The longitude of the recording location.
  \\ \hline
\end{longtable}
\endgroup

\paragraph{Example Response Body} \mbox{}\\[\codeheaderspace]
\begin{jsoncode}
{
  "inputId": "input-1-uuid",
  "path": "./UCF Arboretum/Hurricane Irma/irma-1.wav",
  "site": "UCF Arboretum",
  "series": "Hurricane Irma",
  "recordTimeMs": 1505016000000,
  "coords": {
    "lat": 28.596238,
    "long": -81.191381
  }
}
\end{jsoncode}

\paragraph{Error Handling} \mbox{}\\[\longtableheaderspace]
\begingroup
\renewcommand{\arraystretch}{\cellpaddingvertical}
\begin{longtable}{| m{\errconditioncol} | m{\errcodecol} | m{\errbodycol} |}
  \hline
  \tablehead{Condition}
  & \multicolumn{2}{|l|}{\tablehead{Response}}
  \\ \hline

  A field was included that was not listed as a valid request body field.
  & 400
  & An object containing a single field, \codesnip{message} (String), identifying the invalid field from the request.
  \\ \hline

  A field contains an invalid value.
  & 400
  & An object containing a single field, \codesnip{message} (String), identifying the field in question and its possible values.
  \\ \hline
\end{longtable}
\endgroup
