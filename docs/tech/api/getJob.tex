\subsubsection{Get Job}
Retrieve the details of an existing job by \codesnip{jobId}. This also provides results for any any jobs whose \codesnip{status} is \codesnip{"finished"}.

\paragraph{GET} \mbox{}\\[\codeheaderspace]
\begin{htmlcode}
https://<base-url>/jobs/:jobId
\end{htmlcode}

\paragraph{URL Parameters} \mbox{}\\[\longtableheaderspace]
\begingroup
\renewcommand{\arraystretch}{\cellpaddingvertical}
\begin{longtable}{| m{\fieldcolwidth} | m{\typecolwidth} | m{\desccolwidthlg} |}
  \hline
  \tablehead{Field}
  & \tablehead{Type}
  & \tablehead{Description}
  \\ \hline

  \codesnip{jobId}
  & String
  & The \codesnip{jobId} of the job for which details are desired.
  \\ \hline
\end{longtable}
\endgroup

\paragraph{Example Request: GET} \mbox{}\\[\codeheaderspace]
\begin{htmlcode}
https://<base-url>/jobs/job-1a-uuid
\end{htmlcode}

\paragraph{Response Body Fields} \mbox{}\\[\longtableheaderspace]
\begingroup
\renewcommand{\arraystretch}{\cellpaddingvertical}
\begin{longtable}{| m{\fieldcolwidth} | m{\typecolwidth} | m{\metriccolwidth} | m{\desccolwidthsm} |}
  \hline
  \tablehead{Field}
  & \tablehead{Type}
  & \tablehead{Metric}
  & \tablehead{Description}
  \\ \hline

  \codesnip{jobId}
  & String
  &
  & A unique identifier for the job created.
  \\ \hline

  \codesnip{type}
  & String
  &
  & The type of metric to be run on the job. Possible values: \codesnip{"aci"}, \codesnip{"adi"}, \codesnip{"aei"}, \codesnip{"bi"}, \codesnip{"ndsi"}, \codesnip{"rms"}.
  \\ \hline

  \codesnip{input}
  & String
  &
  & The \codesnip{inputId} of the input to be analyzed for metrics.
  \\ \hline

  \codesnip{spec}
  & String
  &
  & The \codesnip{specId} of the spec to be used for the job.
  \\ \hline

  \codesnip{author}
  & String
  &
  & The \codesnip{userId} of the user who made the job request.
  \\ \hline

  \codesnip{creationTimeMs}
  & Number
  &
  & The time of the job's creation, listed in milliseconds since the Unix epoch.
  \\ \hline

  \codesnip{status}
  & String
  &
  & The status of the job. Possible values: \codesnip{"queued"}, \codesnip{"processing"}, \codesnip{"finished"}, \codesnip{"failed"}, \codesnip{"cancelled"}.
  \\ \hline

  \codesnip{result}
  & Object
  &
  & The data output from processing the job. If the \codesnip{status} of the job is \codesnip{"finished"}, this will be an object. Otherwise, it will be excluded.
  \\ \hline

  \hspace{3mm} \codesnip{aciTotAllL}
  & Number
  & ACI
  & The ACI total for the left channel.
  \\ \hline

  \hspace{3mm} \codesnip{aciTotAllR}
  & Number
  & ACI
  & The ACI total for the right channel.
  \\ \hline

  \hspace{3mm} \codesnip{aciTotAllByMinL}
  & Number
  & ACI
  & The ACI total for the left channel divided by the number of minutes.
  \\ \hline

  \hspace{3mm} \codesnip{aciTotAllByMinR}
  & Number
  & ACI
  & The ACI total for the right channel divided by the number of minutes.
  \\ \hline

  \hspace{3mm} \codesnip{aciFlValsL}
  & Number[]
  & ACI
  & The values of ACI(fl) for the left channel.
  \\ \hline

  \hspace{3mm} \codesnip{aciFlValsR}
  & Number[]
  & ACI
  & The values of ACI(fl) for the right channel.
  \\ \hline

  \hspace{3mm} \codesnip{aciMatrixL}
  & Number[][]
  & ACI
  & The matrix of all values before calculating ACI(fl) for the left channel.
  \\ \hline

  \hspace{3mm} \codesnip{aciMatrixR}
  & Number[][]
  & ACI
  & The matrix of all values before calculating ACI(fl) for the right channel.
  \\ \hline

  \hspace{3mm} \codesnip{adiL}
  & Number
  & ADI
  & The ADI value for the left channel.
  \\ \hline

  \hspace{3mm} \codesnip{adiR}
  & Number
  & ADI
  & The ADI value for the right channel.
  \\ \hline

  \hspace{3mm} \codesnip{bandL}
  & Number[]
  & ADI
  & The vector of proportion values for each band for the left channel.
  \\ \hline

  \hspace{3mm} \codesnip{bandR}
  & Number[]
  & ADI
  & The vector of proportion values for each band for the right channel.
  \\ \hline

  \hspace{3mm} \codesnip{bandRangeL}
  & String[]
  & ADI
  & The vector of frequency values for each band for the left channel.
  \\ \hline

  \hspace{3mm} \codesnip{bandRangeR}
  & String[]
  & ADI
  & The vector of frequency values for each band for the right channel.
  \\ \hline

  \hspace{3mm} \codesnip{aeiL}
  & Number
  & AEI
  & The AEI for the left channel.
  \\ \hline

  \hspace{3mm} \codesnip{aeiR}
  & Number
  & AEI
  & The AEI for the right channel.
  \\ \hline

  \hspace{3mm} \codesnip{areaL}
  & Number
  & BI
  & The area under the curve for the left channel.
  \\ \hline

  \hspace{3mm} \codesnip{areaR}
  & Number
  & BI
  & The area under the curve for the right channel.
  \\ \hline

  \hspace{3mm} \codesnip{ndsiL}
  & Number
  & NDSI
  & The NDSI value for the left channel.
  \\ \hline

  \hspace{3mm} \codesnip{ndsiR}
  & Number
  & NDSI
  & The NDSI value for the right channel.
  \\ \hline

  \hspace{3mm} \codesnip{biophonyL}
  & Number
  & NDSI
  & The biophony value for the left channel.
  \\ \hline

  \hspace{3mm} \codesnip{biophonyR}
  & Number
  & NDSI
  & The biophony value for the right channel.
  \\ \hline

  \hspace{3mm} \codesnip{anthrophonyL}
  & Number
  & NDSI
  & The anthrophony value for the left channel.
  \\ \hline

  \hspace{3mm} \codesnip{anthrophonyR}
  & Number
  & NDSI
  & The anthrophony value for the right channel.
  \\ \hline
\end{longtable}
\endgroup

\paragraph{Example Response Body} \mbox{}\\[\codeheaderspace]
\begin{jsoncode}
{
  "jobId": "job-1a-uuid",
  "type": "aci",
  "input": "input-1-uuid",
  "spec": "spec-a-uuid",
  "author": "user-1-uuid",
  "creationTimeMs": 1546318800000,
  "status": "processing"
}
\end{jsoncode}

\paragraph{Error Handling} \mbox{}\\[\longtableheaderspace]
\begingroup
\renewcommand{\arraystretch}{\cellpaddingvertical}
\begin{longtable}{| m{\errconditioncol} | m{\errcodecol} | m{\errbodycol} |}
  \hline
  \tablehead{Condition}
  & \multicolumn{2}{|l|}{\tablehead{Response}}
  \\ \hline

  The requested job does not exist.
  & 404
  & An object containing a single field, \codesnip{message} (String), stating that a job with the \codesnip{jobId} provided in the request was not found.
  \\ \hline
\end{longtable}
\endgroup
